\documentclass[thesis]{thesis}
%%%% Encodings

\usepackage[utf8]{inputenc} % encoding
\usepackage[english]{babel} % use special characters and also translates some elements within the document.

\usepackage{parskip}        % \par starts on left (not idented)

\usepackage[numbers]{natbib}

% Adds the bibliography, table of figures to the ToC.
\usepackage[nottoc]{tocbibind}

% Hyperlinks \url{url} or \href{url}{name}
\usepackage[colorlinks = true,
            linkcolor = blue,
            urlcolor  = blue,
            citecolor = blue,
            anchorcolor = blue]{hyperref}

% Geometry of the whole document: https://texdoc.org/serve/geometry.pdf/0
\usepackage[margin=1in,headsep=.2in]{geometry}

% \usepackage[document]{ragged2e}  % Left-aligned (whole document)
% \begin{...} ... \end{...}   flushleft, flushright, center

%%%% Standalone

% Allows to compile chapters individually
% https://texdoc.org/serve/standalone.pdf/0   Chapter 5

% \usepackage{standalone}

%%%% Abstract

\usepackage{abstract}

%%%% Math

% Some packages depend on amsmath like semantic

\usepackage{amsmath}
\usepackage{amssymb} % \mathbb{N}
\usepackage{bm} % $\bm{D + C}$
\usepackage{stmaryrd} % \llbracket

\usepackage{amsthm} % \newtheorem, \proof, etc
% \begin{theorem}\label{t:label}  ...  \end{theorem}
% \begin{proof} ... \end{proof}

\theoremstyle{plain} % default
\newtheorem{theorem}{Theorem}[section]
\newtheorem{lemma}[theorem]{Lemma}
\newtheorem*{corollary}{Corollary}

\theoremstyle{definition}
\newtheorem{definition}{Definition}[section]
\newtheorem{conjecture}{Conjecture}[section]
\newtheorem{example}{Example}[section]

\theoremstyle{remark}
\newtheorem*{remark}{Remark}
\newtheorem*{note}{Note}
\newtheorem{case}{Case}

% Defines a new environment to write your or claim - proof
\newenvironment{claim}[1]{\par\noindent\underline{Claim:}\space#1}{}
\newenvironment{claimproof}[1]{\par\noindent\underline{Proof:}\space#1}{\hfill $\blacksquare$}

%%%% Graphics

\usepackage{graphicx}
\graphicspath{{./figures/}}

% \begin{figure}[h]
%   \centering
%   \includegraphics[scale=0.5]{cat}  % [width=\textwidth, height=4cm],
%   \caption{Example of a cat}
%   \label{fig:cat}
% \end{figure}

%%%% Tikz

\usepackage{tikz}
\usetikzlibrary{external, automata, arrows.meta, positioning, calc}
\tikzexternalize[shell escape=-shell-escape,prefix=tikz/cache/]

\newcommand{\inputtikz}[1]{%
  \tikzsetnextfilename{#1}%
  \input{tikz/#1.tex}
}

%%%% Tables
\usepackage{booktabs} % better default table

% \begin{table}
% \centering
% \caption{This is my table, there are many like it, but this one is mine.}
% \label{tbl:mytable}
% \begin{tabular}{llr}
% \toprule
% \multicolumn{2}{c}{Item} \\
% \cmidrule(r){1-2}
% Animal & Description & Price (\$) \\
% \midrule
% Gnat  & per gram & 13.65 \\
%       & each     &  0.01 \\
% Gnu   & stuffed  & 92.50 \\
% Emu   & stuffed  & 33.33 \\
% Armadillo & frozen & 8.99 \\
% \bottomrule
% \end{tabular}
% \end{table}

%%%% Code/Pseudo-code

\newcommand{\code}[1]{\texttt{#1}} % \code{foo.hs} environment


\usepackage{minted}
\usemintedstyle{default}
\newminted{scala}{frame=lines, framerule=2pt}
% \mint{html}|<h2>Something <b>here</b></h2>|
% \inputminted{octave}{BitXorMatrix.m}

%\begin{listing}[H]
  %\begin{minted}[xleftmargin=20pt,linenos,bgcolor=codegray]{haskell}
  %\end{minted}
  %\caption{Example of a listing.}
  %\label{lst:example} % You can reference it by \ref{lst:example}
%\end{listing}


\usepackage[vlined,ruled]{algorithm2e} % pseudo-code
% \newcommand\commentfont[1]{\footnotesize\ttfamily\colorbox{gray}{\textcolor{white}{#1}}}
\newcommand\commentfont[1]{\small\ttfamily\textcolor{darkgray}{#1}}
\SetCommentSty{commentfont}

%%%% Colors

\usepackage{xcolor} % \definecolor, \color{codegray}
\definecolor{codegray}{rgb}{0.9, 0.9, 0.9}
% \color{codegray} ... ...
% \textcolor{red}{easily}

%%%% Glossaries

%\makeglossaries % before entries

%\newglossaryentry{latex}{
    %name=latex,
    %description={Is a mark up language specially suited
    %for scientific documents}
%}

% Referencing a glossary \gls{latex}
% Print glossaries \printglossaries

%%%% Better enumerate

\usepackage{enumitem} % \begin{enumerate}[label=(\alph*)]

%%%% Appendix

\usepackage[titletoc]{appendix}

%%%% Glossaries & Acronyms

\usepackage{glossaries}

\glsdisablehyper % remove hyperlinks to glossaries
\setacronymstyle{long-short} %short-long

\newacronym{tecs}{tECS}{\emph{timed Enumerable Compact Set}}
\newacronym{core}{CORE}{Complex Event Recognition Engine}
\newacronym{mmde}{\textsc{MMDE}}{\emph{Maximal Match Disjoint Enumeration}}
\newacronym{dte}{\textsc{DTE}}{\emph{Distributed tECS Enumeration}}
\newacronym{cea}{CEA}{Complex Event Automata}
\newacronym{rmi}{RMI}{Remote Method Invocation}
\newacronym{dag}{DAG}{Directed Acyclic Graph}
\newacronym{cer}{CER}{Complex Event Recognition}
\newacronym{cep}{CEP}{Complex Event Processing}
\newacronym{cel}{CEL}{\emph{Complex Event Logic}}
\newacronym{socel}{SO-CEL}{Set-Oriented Complex Event Logic}

%%%%%% Subfigure

\usepackage{subcaption} % \subfigure
\usepackage{pgffor} % \foreach
\usepackage{fp}

\def\makeTECS#1#2{%
  \foreach \index in {0, ..., #2} {%
    \FPset\x{#2}
    \FPeval\y{1.0/(x + 1.0) - 0.05}
    \begin{subfigure}[b]{\y\textwidth}
      \centering
      \inputtikz{#1_\index}
      \caption*{S[\index]}
    \end{subfigure}
  }}

%%%% New commands

\newcommand{\comment}[1]{} % \comment{this will not appear in the document}
\newcommand{\cea}{CEA $\mathcal{A} = (Q, \Delta, q_{0}, F)$}
\newcommand{\enumCEA}{${\llbracket \mathcal{A} \rrbracket}^{\epsilon}_{j}(S)$}
\newcommand{\enumNode}{${\llbracket \text{n} \rrbracket}^{\epsilon}_{\mathcal{E}}(j)$}
\newcommand{\tecs}{${\mathcal{E}}$}
\newcommand{\InDoubleBrackets}[1]{\llbracket #1 \rrbracket}


\title{Distributed Complex Event Recognition}
\setthesistype{Master}
\author{Arnau Abella}
\degree{%
  Master in Innovation and Research in Informatics\\
  Advanced Computing
}
\supervisor{%
  \\
  Sergi Nadal, Universitat Politècnica de Catalunya\\
  Stijn Vansummeren, UHasselt – Hasselt University
}

\date{\today}

\begin{document}

\maketitle

\license

\acknowledgements{%
  I have so many people to thank.
  First, and foremost, I am grateful to my advisor Sergi Nadal for his enthusiasm and guidance. This thesis would have not been possible without him. Second, I am due to my co-advisor Stijn Vansummeren. His insightful remarks and suggestions have made some of the contributions of this thesis possible.
  Thanks to Marco Bucchi, Alejandro Grez, Andr\'es Quintana, and Cristian Riveros, among many others, for their numerous contributions to the field of complex event recognition that made this work possible.
  I owe a debt of gratitude to my colleague and friend, Juan Pablo Royo. He is the reason I started this master's degree after three years away from the academia.
  I cannot express enough gratitude toward my family, old and new, who have supported me in every possible way during these two harsh years. To my parents. They have supported me during the span of this master's degree. They are both remarkable parents and I am very lucky to be their son. To my brother Adri\`a. He is the best brother someone could ask for.
  I am left to thank my partner Marta. She has been with me every step on the way. No words can truly express how I feel: I love you and thank you.
}

\abstract{%
  Complex Event Recognition (CER) has emerged as a prominent technology for detecting situations of interest, in the form of query patterns, over large streams of data in real-time. Thus, having query evaluation mechanisms that minimize latency is a shared desiderata. Nonetheless, the evaluation of CER queries is well known to be computationally expensive. Indeed, such evaluation requires the maintenance of a set of partial matches which grows super-linearly in the number of processed events. While most prominent solutions for CER run in a centralized setting, this has proved inefficient for Big Data requirements, where it is necessary to scale the system to cope with an increasing arrival rate of events while maintaining the throughput. To overcome these issues, we propose a novel distributed CER system that focuses on the efficient evaluation of a large class of complex event queries, including $n$-ary predicates, time windows, and partition-by event correlation operator. This system uses a state-of-the-art automaton-based distributed algorithm that circumvents the super-linear partial match problem. Moreover, in the presence of heavy workloads, the system can scale-out by increasing the number of processing units with little overhead. We additionally provide a proof of correctness of the algorithm. We experimentally compare our system against the state-of-the-art sequential CER engine that inspired our work and show that our system outperform its predecessor in the presence of queries with complex predicates. Furthermore, we show that, in the presence of Big Data requirements, our system performance is overall better.
}

\rhead{\thepage}
\pagestyle{fancy}
\tableofcontents
\listoffigures
% \listoftables
% \listofalgorithms
% \addcontentsline{toc}{chapter}{List of Algorithm}
% \listoflistings
% \addcontentsline{toc}{chapter}{List of Source Code}
\mainmatter

\chapter{Introduction}\label{chapter:introduction}

\emph{Complex Event Recognition (CER)} refers to the identification of sets of events that together satisfy some pattern in high-throughput streams of data. This set of events are known as \emph{complex events}. Conceptually, CER systems not only allow to express patterns in terms of the content of the events (like regular stream processing systems), but also in terms of \emph{spatio-temporal constraints}, e.g. the position and the order of the events in the stream. In order to express this spatio-temporal constraints, CER queries include \emph{regular expressions operators} like \emph{unions}, \emph{concatenations} and \emph{kleene stars}.

In recent years, CER has been successfully applied in scenarios like trends on social webs \cite{survey-systems-1}, traffic and transport incidents in smart cities \cite{survey-systems-1}, and real-time analytics \cite{real-time-analytics}. Prominent examples of CER systems from academia and industry include CORE \cite{core}, FlinkCEP \cite{flink-cep}, SASE \cite{sase}, and TESLA \cite{tesla}, among others.  All such systems share the common goal of providing timely reaction to situations of interest in a real-time manner. Thus, having query evaluation mechanisms that minimize latency is a shared desiderata. Nonetheless, the evaluation of CER queries is well-known to be computationally expensive. We illustrate this with the following example.

\begin{example}\label{example:1}
Consider a stream produced by wireless sensors placed in a warehouse, whose main objective is to detect fires. We assume each sensor can measure both temperature (in Celsius degrees) and relative humidity (as a percentage). Additionally, each sensor is assigned a id corresponding to the zone of the warehouse where the sensor is located. The \emph{events} produced by the sensors are composed of the id of the sensor and a measurement corresponding to temperature or relative humidity. We write $T(id, val)$ for an event reporting temperature $val$ from sensor $id$, and $H(id, val)$ for an event reporting humidity $val$ from sensor $id$. An excerpt of the stream of events, indexed by order of arrival, is depicted in Figure~\ref{fig:stream}.

\begin{figure}[H]
  \centering
  \begin{tabular}{|c|c|c|c|c|c|c|c|c|c|c}\hline
    type  &$H$&$T$&$H$&$H$&$T$&$H$&$H$&$T$&$T$ & \ldots \\ \hline
    id  & 1 & 1 & 2 & 1 & 2 & 2 & 1 & 1 & 1 & \multirow{2}{*}{\ldots} \\
    val & 50 & 24& 49& 24& 24& 42& 23& 40& 45\\ \hline
    timestamp & 0 & 1 & 2 & 3 & 4 & 5 & 6 & 7 & 8 & \ldots \\ \hline
  \end{tabular}
  \caption{Exemplary stream of events measuring temperature ($T$) and relative humidity ($H$)}
  \label{fig:stream}
\end{figure}

For the sake of illustration, assume that it has been detected that when the temperature of a storage room increases from below 30 celsius degrees to above 40 celsius degrees and the humidity is below 25\% there is a high probability of fire. The following query retrieves the id of the zone where the fire might be originated so the notification system can warn the security team.

\begin{figure}[h!]
  \begin{minted}[xleftmargin=100pt, linenos=false, fontsize=\footnotesize]{text}
    SELECT t2.id FROM warehouse
    WHERE (T as t1; H as h1; T as t2)
    FILTER t1[val < 30] AND h1[val < 25]
      AND t2[val > 40] AND t1[id] = h1[id]
      AND h1[id] = t2[id]
    WITHIN 10 events
  \end{minted}
  \caption{Query on a wireless sensors network stream, which goal is to detect fires.}
  \label{fig:query:1}
\end{figure}

When the query from Figure~\ref{fig:query:1} is applied to the input stream from Figure~\ref{fig:stream}, the resulting complex events are: $\{ 1, 3, 7 \}$, $\{ 1, 6, 7 \}$, $\{ 1, 3, 6, 7 \}$, $\{ 1, 3, 8 \}$, $\{ 1, 6, 8 \}$, $\{ 1, 3, 6, 8 \}$, $\{ 1, 3, 7, 8\}$, $\{ 1, 6, 7, 8\}$, and $\{ 1, 3, 6, 7, 8\}$. Observe that, within a given time window, the number of \emph{partial matches} that consist of a temperature measurement followed by a humidity measurement followed by a temperature measurement may easily be cubic in the number of events in the window. This gets worsened under the default \emph{skip-till-any-match} \cite{skip-till-any-match} policy, where the set of partial matches can grow \emph{exponentially} in the length of the stream.
\end{example}

In order to overcome the issue illustrated by Example~\ref{example:1}, current CER systems apply clever optimizations to compute the set of partial matches (e.g., lazily computing the set of partial matches \cite{core}). Nevertheless, all of these system still suffer from overhead super-linear in the length of the stream, and thus their scalability is limited to queries over short time windows.

An attempt to overcome the detrimental super-linear complexity of contemporary CER systems is the \emph{COmplex event Recognition Engine (CORE)} engine \cite{core}. Such engine builds on top of a \emph{rigorous} and \emph{efficient} framework for CER that leverages the so called \emph{Complex Event Logic} (CEL) \cite{formal-framework-cep, formal-framework-cer}. To do so, it employs a formal language for specifying complex events, called \emph{CEQL}, that contains many features used in the literature including time windows as well as a partition-by event correlation operator \cite{on-the-expressiveness, core}. Such language can be compiled into a \emph{formal computational model} called \emph{Complex Event Automata} (CEA). CORE incorporates an efficient algorithm for evaluating CEA over event streams using constant time, under data complexity, per event followed by output-linear delay enumeration of the complex events, which is not affected by the length of the stream, size of the query, or size of the time window \cite{formal-framework-cer, core}.

One downside of CORE is that its filtering capabilities are limited to unary predicates. \cite{on-the-expressiveness} shows that unary CEL and CEA are expressively equivalent, however, incomparable when equipped with $n$-ary predicates (e.g., equi-joins like \code{t1[id] = h1[id]}). In particular, when CEL is restricted to binary predicates, it is strictly more expressive than CEA. As a result, CORE cannot embed the processing of $n$-ary filtering predicate in the automaton computational model, and thus cannot guarantee optimal performance under non-unary predicates. This only get aggravated in the presence of iteration operators (i.e., the \emph{kleene star}), where the set of partial matches may grow exponentially in the size of the stream, resulting in an exponential cost of enumerating the complex events.

Departing from the discussion and challenges identified above, in this thesis, we embark on the task of giving a new distributed framework for CER that deals with the limitations of many CER system to express and process complex predicates while preserving optimal performance. To that end, we explore how the evaluation of CER queries with $n$-ary filter predicates can be distributed and parallelized. Considering the fact that such kind of complex filter predicates cannot be embedded into the automaton computational model of CORE, we propose to consider them as a post-process after the enumeration phase. Hence, this thesis is focused on studying and proposing different distribution strategies that optimize such phase. We consider, implement and compare multiple distributed architectures, from the processing of complex events in a centralized fashion distributing the filtering predicates to performing the processing of complex events in a distributed fashion as well. All such features are implemented in a novel distributed architecture for CER, namely DCORE (which stands for \emph{Distributed COmplex event Recognition Engine}).

\textbf{Note.} Throughout the development of this thesis, several new publications on the area had been published (e.g., \cite{formal-framework-cer, core}), which impacted the results of this work.

\section{Contributions}\label{sec:contribution}

Our contributions are summarized as follows:

\begin{enumerate}[label=(\roman*)]
  \item We present a distributed framework for CER. This framework circumvents the filtering limitations of CORE while preserving optimal throughput. Based on this framework, we implemented two different architectures: DCERE and DCORE. DCORE uses the novel distributed evaluation algorithm for CER presented in this work.

  \item We present a novel distributed evaluation algorithm for CER. The proposed algorithm tackles (1) the super-linear complexity of non-unary predicates, and (2) the exponential complexity of the enumeration phase. Our work includes a proof of correctness of this algorithm.

  \item We show that our distributed framework is practical. Our experiments show that, in the presence of Big Data requirements, our distributed framework outperforms CORE on processing queries with complex predicates.
\end{enumerate}

\section{Outline}
\label{sec:outline}

The document is organised as follows. We discuss related work in Chapter~\ref{chapter:related_work}. We give an introduction to CEQL and describe how CEQL is compiled into CEA in Chapter~\ref{chapter:preliminaries}. We introduce the distributed CER framework on Chapter~\ref{chapter:distributed-cer}. In Chapter~\ref{chapter:algorithm} we present the novel distributed evaluation algorithm. We dedicate Chapter~\ref{chapter:experimental_evaluation} to the implementation of the framework and the experiments. We present our conclusions and future work on Chapter~\ref{chapter:conclusion}.

\chapter{Preliminaries}\label{chapter:preliminaries}

% TODO move this to the first time constant-delay is mentioned.
The notion of constant-delay enumeration was defined in the database community \cite{Segoufin13enumeratingwith, 10.1007/978-3-540-74915-8_18} for defining efficiency whenever generating the output might use considerable time. An enumeration is performed in constant-delay if it takes constant time between any two consecutive solutions.

% Same for data complexity
There are three ways to measure the complexity of evaluating queries in a specific language. First, one can fix a specific query in the language and study the complexity of applying this query to arbitrary databases. The complexity is then given as a function of the size of the databases. We call this complexity \textit{data complexity} \cite{DataComplexity}.

\section{CORE}

\begin{definition}[Match]
  \label{def:match}
  TODO
\end{definition}

\begin{definition}[Maximal Match]
  \label{def:maximalmatch}
  TODO
\end{definition}

\section{SO-CEL}

\section{Chapter summary}

\chapter{Preliminaries}\label{chapter:preliminaries}

In this section, we introduce the formal background that support our study.
First, we introduce \emph{distributed computing}.
Secondly, we describe CEQL and give a formal description.
Thirdly, we briefly discuss \emph{selection strategies}.
Lastly, we introduce the computational model CEA and how to compile unary CEQL to CEA.

\section{Distributed computing}\label{sec:distributed_computing}

A \emph{distributed system} is a system whose components are located on different networked computers that communicate to each other by message passing in order to achieve a common goal. The main three characteristics of a distributed system are: concurrency of the components, lack of global memory and clock, and tolerance to failure of individual components \cite{distributed-computing-book}. Nowadays, the term is used in a much wider sense, even referring to autonomous processes that run on the same physical computer and interact with each other by message passing. In our work, we do not make a distinction on whether the system operators on different networked computers or the same physical computer.

A \emph{distributed program} is composed of an ordered-set of $n$ asynchronous processes $\mathcal{P} = \{ p_{1}, p_{2}, \ldots, p_{n}\}$. For a process $p_{i}$ with $1 \le i \le n$, define its \emph{index}, denoted index($p_{i}$), as index($p_{i}$) = $i \in \mathbb{N}$. The index of a process can be used as a \emph{unique} identifier. The processes do not share a global memory and communicate solely by passing messages. Process execution and message transfer are asynchronous. Without loss of generality, we assume that each process is running on a different processor. Let $C_{ij}$ denote the channel from process $p_{i}$ to process $p_{j}$ and let $m_{ij}$ denote a message sent by $p_{i}$ to $p_{j}$. The message transmission delay is finite and unpredictable \cite{distributed-computing-book}.

\section{Complex event logic}\label{sec:ceql}

In this section, we introduce CEQL by means of an example, and then, we give the formal syntax and semantics of CEQL.

\emph{Complex Event Query language (CEQL)} is a practical CER language based on \emph{Complex Event Logic (CEL)}, which is a formal logic that is built from the common operators in the literature of CER and whose expressiveness and complexity have been in-depth studied in \cite{formal-framework-cep,on-the-expressiveness,formal-framework-cer}.

We continue



\begin{figure}[H]
  \begin{minted}[xleftmargin=100pt, linenos=false, fontsize=\scriptsize]{text}
    SELECT *
    FROM S
    WHERE (T as t1; T+ as ts)
    FILTER t1[tmp < 30]
      AND ts[tmp > 30]
    PARTITION BY id
    WITHIN 5 minutes
  \end{minted}
  \caption{CEQL query on a synthetic stream S.}
  \label{fig:query}
\end{figure}

\section{Selection strategies}\label{sec:selection_strategies}

In order to speed up the matching process, it is common to restrict the set of results \cite{selection-strategies-literature-1,selection-strategies-literature-2,selection-strategies-literature-3}. Sadly, most proposals in the literature introduce heuristics to a particular computational model without describing how the semantics are affected. In \cite{formal-framework-cer}, a more general approach, \emph{selection strategies} as unary operators over core-CEL formulas, was introduced. Four selection strategies called strict (\textsc{strict}), next (\textsc{nxt}), last (\textsc{last}) and max (\textsc{max}) were formally defined.

\section{Computational model}\label{sec:cea}

\begin{figure}[H]
  \centering
  \begin{subfigure}[b]{\textwidth}
    \centering
    \inputtikz{cea}
    \vspace*{2em}
  \end{subfigure}
  \begin{subfigure}[t]{\textwidth}
    \centering
    \inputtikz{stream}
  \end{subfigure}

  \caption{A CEA representing the query from Figure~\ref{fig:query} and an example of stream.}
  \label{fig:cea}
\end{figure}

\chapter{???}\label{chapter:???}

%%%%%%%%%%%%%%%%%%%%%%%%%%%%%%%%%%%%%%%%%%%%%%%%%%%%%%%%%%%%

\section{Distribution strategies}\label{sec:distribution_strategies}
% I like how LoadBalancingForSkewedStreamsOnHeterogeneousClusters is organized

A distributed program is composed of an ordered-set of $n$ asynchronous processes $\mathcal{P} = \{ p_{1}, p_{2}, \ldots, p_{i}, \ldots, p_{n}\}$. For a process $p_{i}$, define its \emph{index}, denoted index($p_{i}$), as index($p_{i}$) = $i \in \mathcal{N}$. The processes do not share a global memory and communicate solely by passing messages. Process execution and message transfer are asynchronous. Without loss of generality, we assume that each process is running on a different processor. Let $C_{ij}$ denote the channel from process $p_{i}$ to process $p_{j}$ and let $m_{ij}$ denote a message sent by $p_{i}$ to $p_{j}$. The message transmission delay is finite and unpredictable.

%%%%%%%%%%%%%%%%%%%%%%%%%%%%%%%%%%%%%%%%%%%%%%%%%%%%%%%%%%%%

\section{Distributed enumeration of complex events}
\label{sec:distributed_enumeration_of_the_complex_events}

In this section, we present two distributed enumeration algorithms: (1) \acrfull{mmde}, and (2) \acrfull{dte}.

CORE's evaluation algorithm guarantees, under data complexity, constant time per event and constant-delay enumeration of the output \cite{core}. However, under the default \emph{skip-till-any-match} \cite{skip-till-any-match} policy in CORE, non-contiguous sequencing and iteration can cause the amount of complex events to grow exponentially in the size of the stream \cite{formal-framework-cer}. The evaluation algorithms needs to materialize the set of partial matches each time an enumeration is required; therefore, in the worst case, enumerating all complex events generated by an event is exponential in the length of the stream. In order to deal with the exponential complexity of materializing and enumerating complex events in the evaluation algorithm, we propose to employ distributed execution of the process \cite{distributed-computing-book}.

\subsection{Maximal Matches Disjoint Enumeration}\label{subsec:mmde}

We propose a novel enumeration algorithm called \acrfull{mmde}

Talk about the algorithm: build on top of CORE, in the compilation of the query,...

Define: match, maximal match, configuration, ... (get inspire by CORE's paper)

Define the operations on the tree.

Define group by (maybe, it would be easier to define a new concept)

\begin{algorithm}[H]
  \DontPrintSemicolon
  \SetAlgoNoEnd % don't print end
  \SetAlgoNoLine % no vertical lines
  \LinesNumbered
  \SetKwProg{Procedure}{procedure}{}{}
  \SetKwFunction{MMDE}{\textsc{MaximalMatchesDisjointEnumeration}}
  \SetKwFunction{Enumerate}{\textsc{Enumerate}}

  \Procedure{\MMDE{$M$, $W$}}{
    \KwIn{A set of maximal matches $M := \{M_{1}, \ldots, M_{n}\}$ \newline
      and a set of workers $W := \{w_{1},\ldots, w_{m}\}$.
    }
    \KwResult{Enumerates all \emph{submatches} $\subseteq M$ without repetitions.}
    $C \leftarrow \emptyset$\;
    \ForEach{$M_{i} \in M$}{
        $C \leftarrow C \cup \textsc{Configurations}(M_{i}).map(\lambda c \to ( c, M_{i} ))$\;
    }
    $D \leftarrow C.groupBy(\lambda (c, \_ ) \to c)$\;
    $\textsc{Distribute}(W, D)$
  }
  \;
\caption{Non-repeated enumeration of a set of maximal matches.}
\label{algo:mmde}
\end{algorithm}

% This procedure enumerates all submatches of M without repetitions.
% It stills enumerates all submatches but only outputs non-repeated.
% It efficiently detects repetitions by constructing an n-ary tree of complex events.
% The complexity is still exponential w.r.t. the size of the largest iteration.
% The exponential time enumeration must be repeated a constant factor of times.
\begin{algorithm}[H]
  \DontPrintSemicolon
  \SetAlgoNoEnd % don't print end
  \SetAlgoNoLine % no vertical lines
  \LinesNumbered
  \SetKwProg{Procedure}{procedure}{}{}
  \SetKwFunction{Enumerate}{\textsc{Enumerate}}
  \SetKwFunction{Enumeratee}{\textsc{Enumerate'}}

  \Procedure{\Enumerate{}}{
    \KwData{A set of tuples $A = \{ (c, \{ M_{1}, \ldots, M_{n}\}) \}$ where $c$ is a \emph{configuration} and $M_{i}$ are maximal matches.}
    \KwOut{The set of all submatches without repetitions.}
    \ForEach{$(c, M) \in A$}{
      $T \leftarrow$ \text{new-root()}
      \ForEach{$M_{i} \in M$}{
        $G \leftarrow \textsc{GroupBy}(M_{i})$\;
        $\textsc{Enumerate'}(T, G, \emptyset, \bot)$\;
        }
    }
  }
  \;
  \Procedure{\Enumeratee{$n, G, S, new$}}{
    \KwData{A node $n$, a set of grouped events $G$, a time-ordered set of events $S$, and a boolean $new$.}
    \Switch{$G$}{
      \uCase{$\emptyset$}{
        \If{$new$}{
          \Return{$S$}
        }
      }
      \uCase{$g \cup G'$}{
        $k \leftarrow c(g.type)$
        $E \leftarrow \binom{g}{k}$
        \ForEach{$e \in E$}{
          \eIf{$\exists n' \in n.children \land n'.event = e$}{
            $\textsc{Enumerate'}(n', G', S \cup e, new)$\;
          }{
            $p \leftarrow$ new-node($e$)\;
            $n.children.add(p)$\;
            \textsc{Enumerate'}$(p, G', S \cup e, \top)$\;
          }
        }
      }
    }
  }
  \;
\caption{Non-repeated enumeration of a set of maximal matches given a predicate configuration.}
\label{algo:enumerate}
\end{algorithm}

\begin{algorithm}[H]
  \DontPrintSemicolon
  \SetAlgoNoEnd % don't print end
  \SetAlgoNoLine % no vertical lines
  \LinesNumbered
  \SetKwProg{Procedure}{procedure}{}{}
  \SetKwFunction{Configurations}{\textsc{Configurations}}

  \Procedure{\Configurations{$M$}}{
    \KwIn{A match $M = \{e_{1}, \ldots, e_{n}\}$ where $e_{i}$ is an event of type $t \in T$.}
    \KwOut{A set $C$ of configurations $c := T \times \mathbb{N}$ where $c$ is the mapping from the event type $t \in T$ to the size of the iteration of the event type $t$ in the submatches of $M$.}
    $V \leftarrow newList$\;
    $e_{0} \cup M' \leftarrow pop(M)$\;
    $A \leftarrow \{ e_{0} \}$\;
    $A.type \leftarrow e_{0}.type$\;
    \For{event $e$ in $M'$}{
      \eIf{$e.type = A.type$}{
        $A \leftarrow A \cup e$\;
        \uIf{$isLast(e)$} {
          $V \leftarrow V + enumFromTo(1, |A|)$
        }
      }{
        $V \leftarrow V + enumFromTo(1, |A|)$\;
        $A \leftarrow \{ e \}$\;
        $A.type \leftarrow e.type$\;
      }
    }
    $WW \leftarrow V_{1} \times \cdots \times V_{n}$\tcp*[l]{$V = \{V_{1}, \cdots, V_{n}\}$}
    $T \leftarrow types(M)$\tcp*[l]{Ordered set of types e.g. $types(A_{1}A_{2}B_{1}C_{1}) = \{A,B,C\}$}
    $C \leftarrow \emptyset$\;
    \ForEach(\tcp*[h]{E.g. $W = \{1, 2, 1\}$}){$W \in WW$}{
      $c \leftarrow \emptyset$\tcp*[l]{E.g. $c = \{(A,1), (B,2), (C, 1)\}$}
      \For{$i \leftarrow 1$ \KwTo $|W|$}{
        $c \leftarrow c \cup (T[i], W[i])$\;
      }
      $C \leftarrow C \cup c$\;
    }
    \Return{C}
  }
\caption{Computes all disjoint configurations of a maximal match.}
\label{algo:configurations}
\end{algorithm}

% You need to make the following observations of "Maximal Matches Enumeration":
% 1. The algorithm produces disjoint submatches given a maximal match.
% 2. The algorithm produces non-disjoint submatches given multiple maximal matches.

% But (2) can be analyzed further:
% 1. Disjoint configurations produce disjoint submatches.
% 2. Non-disjoint configurations produce non-disjoint submatches.

% From previous observations we can conclude that repeated submatches are only generated by applying the same configuration to different maximal matches.

% Uniqueness of submatches is guaranteed by (3) and (4).
% (3) guarantees that the output of each worker is disjoint wrt the others.
% (4) guarantees that the output of a worker is disjoint.

% The complexity of the algorithm remains the same if we accomplish linear time enumeration in each worker (this is the tricky part).

\begin{lemma}[TODO]
  \label{lemma:todo}
  TODO
\end{lemma}

\begin{theorem}[TODO]
  \label{theorem:todo}
  TODO
\end{theorem}

\begin{example}[TODO]
  \label{example:todo}
  TODO
\end{example}

% Soundness and Completness of an Algorithm
%
% Let S be the set of all right answers.
% A sound algorithm never includes a wrong answer in S, but it might miss a few right answers.
% A complete algorithm should get every right answer in S: include the complete set of right answers. But it might include a few wrong answers.
%
% Careful! Soudness and completness in logic has another meaning https://math.stackexchange.com/questions/105575/what-is-the-difference-between-completeness-and-soundness-in-first-order-logic

%%%%%%%%%%%%%%%%%%%%%%%%%%%%%%%%%%%%%%%%%%%%%%%%%%%%%%%%%%%%

\subsection{Distributed tECS enumeration}\label{subsec:distributed_tecs_enumeration}

In this section, we propose \acrfull{dte}, a novel extension of the evaluation algorithm in \cite{core}. In particular, \acrshort{dte} distributes the potentially exponential workload of \textsc{Enumerate} among $n$ asynchronous processes $p_{1}, \ldots, p_{n}$ while preserving the constant time per input event update of the data structure that compactly represents the set of partial matches and the output-linear delay enumeration of the results.

Before giving the formal description of the algorithm, we need to extend the definition of \acrfull{tecs}, as introduced in \cite{core}. A \acrshort{tecs} is a \acrfull{dag} \tecs with two kinds of nodes; union nodes and non-union nodes. Every union node u has exactly two children, the left child left(u) and the right child right(u). Every non-union node n is labelled by a stream position (an element of $\mathcal{N}$) and has at most one child. If non-union node n has no child it is called a \emph{bottom node}, otherwise it is an \emph{output node}. We write pos(n) for the label of non-union node n and next(o) for the unique child of output node o. For a node n, define its \emph{descending-paths}, denoted paths(n), as follows: if n is a bottom node, then paths(n) = 1; if n is an output node, then paths(n) = paths(next(n)); otherwise, paths(n) = paths(left(n)) + paths(right(n)). The descending-paths can be computed in constant.

A \acrshort{tecs} represents sets of \emph{open} complex events. An \emph{open complex event} is a pair $(i, D)$ where $i \in \mathcal{N}$ and $D$ is a finite subset of $\{i, i+1, \ldots\}$. Intuitively, when processing a stream, the open complex events represented by a tECS are partial results that may later become full complex events. Remember that the purpose of constructing \tecs is to be able to enumerate the set \enumCEA at every $j$. To achieve that goal, it will be necessary to enumerate, for certain nodes n in \tecs, the set $\InDoubleBrackets{\text{n}}^{\epsilon}_{\mathcal{E}}(j) := \{ ([i, j], D) | (i, D) \in \InDoubleBrackets{\text{n}}_{\mathcal{E}} \land j - i \leq \epsilon \}$ i.e. all open complex events represented by n that, when closed with j, are within a time window of size $\epsilon$.

Recall that we imposed three restrictions on the structure of a \acrshort{tecs}: (1) it needs to be \emph{time-ordered}, (2) it needs to be \emph{k-bounded}, and (3) its needs to be \emph{duplicate-free}.

\begin{theorem}[{\cite[Theorem 2]{core}}]\label{theorem:theorem2}
Fix k. For every k-bounded and time-ordered tECS \tecs, and for every duplicate-free node n of \tecs, time-window bound $\epsilon$, and position $j$, the set \enumNode can be enumerated with output-linear delay and without duplicates.
\end{theorem}

To ensure that we may enumerate \enumCEA from \tecs by use of Theorem \ref{theorem:theorem2}, \tecs will always be time-orederd, $k$-bounded for $k = 3$, and duplicate-free.

We defined three operations on \tecs: new-bottom($i$), extend(m, $j$) and union($\text{n}_{1},\text{n}_{2}$). The first method, new-bottom($i$) adds a new bottom node b labelled $i$ to \tecs. The second method, extend(n, $j$) adds a new output node o to \tecs with pos(o) = $j$ and next(o) = n. The third method, union($\text{n}_{1},\text{n}_{2}$) returns a node u such that $\InDoubleBrackets{u}_{\mathcal{E}} = \InDoubleBrackets{\text{n}_{1}}_{\mathcal{E}} \cup \InDoubleBrackets{\text{n}_{2}}_{\mathcal{E}}$. Any \acrshort{tecs} that is created using only these three methods is time-ordered and $3$-bounded.

%%%%%%%%%%%%%%%%%%%%%%%%%%%%%%%%%%%%%%%%%%%

We are ready to description algorithm \acrshort{dte}. An efficient implementation would be execute \cite[Algorithm 1]{core} on a single process e.g. $p_{0}$, and distribute the \acrshort{tecs} \tecs to the rest of the processes $p_{1}, \ldots, p_{n}$, so $p_{0}$ could keep ingesting events at constant time while the rest of processes could be working on the possibly exponential enumeration of \tecs. However, if done in a na\"ive way, distributing the \acrshort{tecs} on each event that triggers a complex event would take time proportional to its size which is at most linear in the length of the stream breaking the constant update time. Hence, the distribution of the tECS has to be done incrementally per input event to preserve the constant time complexity. An algorithm that that incrementally distributes the tECS would need to send to each process all new node added to the tECS at iteration $j$ and additional information on how to add these nodes to the tECS of iteration $j-1$. Furthermore, each process would need to keep an updated hash table, similar to the one used in the evaluation algorithm, in order to apply the incremental changes instructed by the messages from the centralized evaluation process. We argue that this incremental distribution algorithm is \emph{asymptotically equivalent} to executing the update phase from \cite[Algorithm 1]{core} on each process. Both algorithms needs to incrementally update the tECS and the hash table $T$ but the later needs to iterate over all transitions $\Delta$ in the worst-case, however, this takes constant time under data complexity. Consequently, we choose to replicate the update phase of Algorithm 1 on each process and leave for future work a more sophisticated approach.





\begin{algorithm}[H]
  \DontPrintSemicolon
  \SetAlgoNoEnd % don't print end
  \SetAlgoNoLine % no vertical lines
  \SetKwProg{Procedure}{procedure}{}{}
  \SetKwFunction{Enumerate}{\textsc{Enumerate}}
  \Procedure{\Enumerate{$\mathcal{E}, n, \epsilon, j, p_{i}$}}{
    $a \leftarrow \lceil \text{paths(n)} \ / \ {|\mathcal{P}|} \rceil$\;
    $s \leftarrow \text{index}(p_{i}) \cdot a$\;
    st $\leftarrow$ new-stack()\;
    $\tau \leftarrow j - \epsilon $\;
    \If{$\text{max(n)} \ge \tau$}{
      push(st,($n$, $\emptyset$, $s$, $a$))\;
    }
    \While{$(n', P, s', a') \leftarrow$ pop(st)}{
      \While{true}{
        \If{$n' \in N_{B}$}{
          output([pos($n'$), $j$], $P$)\;
          \textbf{break}\;
        }
        \ElseIf{$\text{n}' \in N_{O}$}{
          $P \leftarrow P \ \cup $ {pos($n'$)}\;
          $n' \leftarrow $ next($n'$)\;
        }
        \ElseIf{$n' \in N_{U}$}{
          \If{$max(right(n')) \ge \tau$}{
            \eIf{$paths(left(n')) > s'$}{
              $a'' \leftarrow a' - max(0, paths(left(n)) - s')$\;
            }{
              $a'' \leftarrow a'$\;
            }
            $s'' \leftarrow s' - paths(left(n))$\;
            \If{$paths(right(n)) > s'' \land a'' > 0$}{
              push(st, (right($n'$), $P$, $s''$, $a''$))\;
            }
          }
          \eIf{$paths(left(n')) > s'$}{
            $n' \leftarrow left(n')$\;
          }{
            \textbf{break}\;
          }
        }
      }
    }
  }
\caption{Distributed enumeration of $\InDoubleBrackets{\text{n}}^{\epsilon}_{\mathcal{E}}(j)$}
\label{algo:enumeration}
\end{algorithm}

We provide Algorithm \ref{algo:enumeration} and show that: (1) it enumerates the set \enumNode, (2) it does so with output-linear delay, and (3) it statically distributes the workload among the $n$ processes. In the following sections, we denote the sets of bottom, output and union nodes by $N_{B}$, $N_{O}$ and $N_{U}$, respectively.

TODO: explain algorithm

% CEA AB+
\begin{figure}[H]
  \centering
  \begin{subfigure}[t]{\textwidth}
    \centering
    \inputtikz{streamAB+}
  \end{subfigure}
  \\
  \begin{subfigure}[b]{\textwidth}
    \begin{minted}[xleftmargin=40pt, linenos=false]{text}
      SELECT *
      FROM S
      WHERE A as a; B + as bb
    \end{minted}
  \end{subfigure}
  \\
  \begin{subfigure}[b]{\textwidth}
    \centering
    \inputtikz{ceaAB+}
  \end{subfigure}
  \caption{A CEA representing $Q_{1}$ from Figure 1 and some of its runs on an example stream.}
  \label{fig:label}
\end{figure}

\begin{figure}[H]
  \centering
  \begin{subfigure}[t]{0.1\linewidth}
    \inputtikz{AB+_0}
  \end{subfigure}
  \begin{subfigure}[t]{0.1\linewidth}
    \inputtikz{AB+_1}
  \end{subfigure}
  \begin{subfigure}[t]{0.24\linewidth}
    \inputtikz{AB+_2}
  \end{subfigure}
  \begin{subfigure}[t]{0.24\linewidth}
    \inputtikz{AB+_3}
  \end{subfigure}
  \begin{subfigure}[t]{0.28\linewidth}
    \inputtikz{AB+_4}
  \end{subfigure}
  \caption{Illustration of Algorithm TODO on the CEA $\mathcal{A}$ and stream $S$ of Figure ???.}
  \label{fig:label}
\end{figure}

\begin{figure}[H]
  \centering
  \begin{subfigure}[t]{0.24\linewidth}
    \inputtikz{AB+_enumeration_0}
  \end{subfigure}
  \begin{subfigure}[t]{0.24\linewidth}
    \inputtikz{AB+_enumeration_1}
  \end{subfigure}
  \begin{subfigure}[t]{0.24\linewidth}
    \inputtikz{AB+_enumeration_2}
  \end{subfigure}
  \begin{subfigure}[t]{0.24\linewidth}
    \inputtikz{AB+_enumeration_3}
  \end{subfigure}
  \caption{Illustration of Algorithm \ref{algo:enumeration} on the CEA $\mathcal{A}$ and stream $S$ of Figure ???.}
  \label{fig:label}
\end{figure}

% CEA A+B+
\begin{figure}[H]
  \centering
  \begin{subfigure}[t]{\textwidth}
    \centering
    \inputtikz{streamA+B+}
  \end{subfigure}
  \\
  \begin{subfigure}[b]{\textwidth}
    \begin{minted}[xleftmargin=40pt, linenos=false]{text}
      SELECT *
      FROM S
      WHERE A + as aa; B + as bb
    \end{minted}
  \end{subfigure}
  \\
  \begin{subfigure}[b]{\textwidth}
    \centering
    \inputtikz{ceaA+B+}
  \end{subfigure}
  \caption{A CEA representing $Q_{1}$ from Figure 1 and some of its runs on an example stream.}
  \label{fig:label}
\end{figure}

\begin{figure}[H]
  \begin{subfigure}[t]{0.1\linewidth}
    \inputtikz{A+B+_0}
  \end{subfigure}
  \begin{subfigure}[t]{0.1\linewidth}
    \inputtikz{A+B+_1}
  \end{subfigure}
  \begin{subfigure}[t]{0.24\linewidth}
    \inputtikz{A+B+_2}
  \end{subfigure}
  \begin{subfigure}[t]{0.24\linewidth}
    \inputtikz{A+B+_3}
  \end{subfigure}
  \begin{subfigure}[t]{0.28\linewidth}
    \inputtikz{A+B+_4}
  \end{subfigure}
  \caption{Illustration of Algorithm 1 on the CEA $\mathcal{A}$ and stream $S$ of Figure ???.}
  \label{fig:label}
\end{figure}

%%%%%%%%%%%%%%%%%%%%%%%%%%%%%%%%%%%%%%%%%%%%%%%%%%%%%%%%%%%%

\section{Chapter summary}

TODO

% The title is not very accurate
\chapter{Implementation: ?}\label{implementation}

\section{Chapter summary}

\chapter{The implementation}\label{chapter:implementation}

\chapter{Conclusions and future work}\label{chapter:conclusion}

We presented a novel distributed CER framework that focuses on the efficient evaluation of a large class of complex event queries, including $n$-ary predicates. We proposed two implementations based on such framework: DCERE and DCORE. In particular, DCORE uses a novel evaluation algorithm that tackles the super-linear complexity of non-unary predicates and the exponential complexity of the enumeration. Furthermore, our experiments results show that our framework is practical and outperforms its competitors on queries with complex predicates over large streams of data.

We plan to continue our research in a few directions. We will extend our framework with a generic rewrite and refine algorithm. We are also preparing an extension of the distributed evaluation algorithm that takes into account time windows during the distribution phase. Finally, we will explorer the correlation capabilities offered register and data automata, and the aggregation capabilities of cost register automata.

\chapter{Conclusions and future work}\label{chapter:conclusion}

Lorem ipsum dolor sit amet, consectetur adipiscing elit, sed do eiusmod tempor incididunt ut labore et dolore magna aliqua. Viverra justo nec ultrices dui sapien eget. Lorem donec massa sapien faucibus et molestie ac feugiat sed. Amet venenatis urna cursus eget nunc scelerisque viverra mauris. Quis vel eros donec ac odio tempor. Iaculis nunc sed augue lacus. Fringilla ut morbi tincidunt augue interdum velit euismod. Amet dictum sit amet justo. Pellentesque adipiscing commodo elit at imperdiet dui. Est ante in nibh mauris cursus mattis molestie a. Tortor dignissim convallis aenean et tortor. Praesent tristique magna sit amet. Lectus arcu bibendum at varius vel pharetra vel turpis nunc. Tristique risus nec feugiat in fermentum posuere urna.

Quis lectus nulla at volutpat diam ut. Convallis aenean et tortor at risus. At tempor commodo ullamcorper a lacus vestibulum sed. Vitae justo eget magna fermentum iaculis eu non diam. Pharetra diam sit amet nisl suscipit adipiscing bibendum. Phasellus vestibulum lorem sed risus ultricies tristique nulla. Viverra accumsan in nisl nisi scelerisque eu. Cursus in hac habitasse platea dictumst quisque sagittis purus. Sagittis eu volutpat odio facilisis mauris sit amet. Suscipit tellus mauris a diam maecenas. Vitae proin sagittis nisl rhoncus mattis rhoncus. Tristique et egestas quis ipsum suspendisse. Ipsum dolor sit amet consectetur adipiscing elit ut aliquam purus. Ullamcorper morbi tincidunt ornare massa eget. Nisl pretium fusce id velit ut. Dis parturient montes nascetur ridiculus mus mauris vitae ultricies leo. Enim sit amet venenatis urna cursus.


\begin{appendices}
\chapter{Proofs of Chapter 4}\label{appendix:A}

\section{Proof of Theorem 1}
\label{appendix:A:sec:1}

\end{appendices}

\bibliographystyle{unsrtnat}
\bibliography{bibliography}

\end{document}
