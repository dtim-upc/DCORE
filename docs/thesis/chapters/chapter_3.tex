\chapter{Preliminaries}\label{chapter:preliminaries}

% TODO move this to the first time constant-delay is mentioned.
% In CORE they use output-linear delay instead of constant-delay ...
The notion of constant-delay enumeration was defined in the database community \cite{constant-delay-1, constant-delay-2} for defining efficiency whenever generating the output might use considerable time. An enumeration is performed in constant-delay if it takes constant time between any two consecutive solutions. Formally, it requires the existence of a routine \textsc{Enumerate} that receives $t_{j}$ as input and outputs all complex events in $\llbracket n \rrbracket^{\epsilon}_{\mathcal{E}}(j)$ without repetitions, while spending a constant amount of time before and after each output. Naturally, the time to generate a complex event $C$ must be linear in $|C|$.

% Same for data complexity
There are three ways to measure the complexity of evaluating queries in a specific language. First, one can fix a specific query in the language and study the complexity of applying this query to arbitrary databases. The complexity is then given as a function of the size of the databases. We call this complexity \emph{data complexity} \cite{data-complexity}.


% TODO
In order to speed up the matching process, it is common to restrict the set of results \cite{10.1016/j.scico.2010.06.010, 10.1145/1142473.1142520, Zhang2014OnCA}. Sadly, most proposals in the literature introduce heuristics to a particular computational model without describing how the semantics are affected. In \cite{formal-framework-cer}, a more general approach, \emph{selection strategies} as unary operators over core-CEL formulas, was introduced. Four selection strategies called strict (\textsc{strict}), next (\textsc{nxt}), last (\textsc{last}) and max (\textsc{max}) were formally defined.


A \textbf{distributed program} is composed of an ordered-set of $n$ asynchronous processes $\mathcal{P} = \{ p_{1}, p_{2}, \ldots, p_{i}, \ldots, p_{n}\}$. For a process $p_{i}$, define its \emph{index}, denoted index($p_{i}$), as index($p_{i}$) = $i \in \mathcal{N}$. The processes do not share a global memory and communicate solely by passing messages. Process execution and message transfer are asynchronous. Without loss of generality, we assume that each process is running on a different processor. Let $C_{ij}$ denote the channel from process $p_{i}$ to process $p_{j}$ and let $m_{ij}$ denote a message sent by $p_{i}$ to $p_{j}$. The message transmission delay is finite and unpredictable.
