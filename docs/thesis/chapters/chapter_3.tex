\chapter{Preliminaries}\label{chapter:preliminaries}

In this section, we introduce the formal background that support our study.
First, we introduce the concepts behind \emph{distributed computing}.
Secondly, we describe CEQL and give a formal description.
Thirdly, we briefly discuss \emph{selection strategies}.
Lastly, we introduce the computational model CEA.

\section{Distributed computing}\label{sec:distributed_computing}

A \emph{distributed system} is a system whose components are located on different networked computers that communicate to each other by message passing in order to achieve a common goal. The main three characteristics of a distributed system are: concurrency of the components, lack of global memory and clock, and tolerance to failure of individual components \cite{distributed-computing-book}. Nowadays, the term is used in a much wider sense, even referring to autonomous processes that run on the same physical computer and interact with each other by exchanging messages. In our work, we do not make a distinction on whether the system operates on a cluster of networked computers or in a single multi-core computer.

A \emph{distributed program} is composed of an ordered-set of $n$ asynchronous processes $\mathcal{P} = \{ P_{1}, P_{2}, \ldots, P_{n}\}$. For a process $P_{i}$ with $1 \le i \le n$, define its \emph{index}, denoted index($P_{i}$), as index($P_{i}$) = $i \in \mathbb{N}$. The index of a process can be used as a \emph{unique} identifier. The processes do not share a global memory and communicate solely by passing messages. Process execution and message transfer are asynchronous. Without loss of generality, we assume that each process is running on a different processor. Let $C_{ij}$ denote the channel from process $P_{i}$ to process $P_{j}$ and let $m_{ij}$ denote a message sent by $P_{i}$ to $P_{j}$. The message transmission delay is finite and unpredictable \cite{distributed-computing-book}.

\section{Complex event logic}\label{sec:ceql}

\emph{Complex Event Query Language (CEQL)} is a practical CER language based on \emph{Complex Event Logic (CEL)}, which is a formal logic that is built from the common operators in the literature of CER and whose expressiveness and complexity have been diligently studied in \cite{formal-framework-cep,on-the-expressiveness,formal-framework-cer}. We introduce the most relevant features of CEQL by means of an example.

\begin{example}
We retake the previous example which aims on the detection of fires in a stream produced by a network of wireless sensors placed in a warehouse. Suppose that we are interested in all $n$-tuples of \code{T} events where the first has temperature below $30$ Celsius degrees and the rest has temperature above $30$ Celsius degrees, partitioned by the zone of the warehouse where the sensor is placed. The query from Figure~\ref{fig:query:2} expresses this in CEQL.

\begin{figure}[H]
  \begin{minted}[xleftmargin=120pt, linenos=false, fontsize=\footnotesize]{text}
    SELECT *
    FROM warehouse
    WHERE (T as t1; T+ as ts)
    FILTER t1[val < 30]
      AND ts[val > 30]
    PARTITION BY id
    WITHIN 5 minutes
  \end{minted}
  \caption{CEQL query on a wireless sensors network stream.}
  \label{fig:query:2}
\end{figure}

The \code{FROM} clause indicates the input streams. The \code{WHERE} clause indicates the pattern of events that need to be matched in the stream. The pattern can be any \emph{unary} CEL expression \cite{formal-framework-cer}. In our query, the pattern \code{T as t1; T+ as ts} indicates that we want to capture all complex events that consist of an event type \code{T} followed by an arbitrary number of events of type \code{T}. In particular, the operator (;) indicates sequencing and the operator (+) indicates non-empty repeated sequencing. We want to remark that sequencing in CEL is non-contiguous. Consequently, the \code{T} events do not need to be contiguous --- there might be other events in between. The \code{FILTER} clause allows to filter events by unary predicates. The clause \code{PARTITION BY} allows to express correlation among events in the form of equi-joins. The \code{WITHIN} clause specifies the time-window. In our query, the time between the first event \code{T} and the last event \code{T} must be within 5 minutes. Finally, the \code{SELECT} clause allows to project the result.
\end{example}

Next, we give the formal syntax and semantics of CEQL.

\textbf{Events, complex events, and valuations}. Fix a set of \emph{event types} \textbf{T} (e.g., \code{H} and \code{T}), a set of \emph{attribute names} \textbf{A} (e.g., $id$, and $val$), and a set of \emph{data values} \textbf{D} (e.g., integer values, string values, etc.). A \emph{data-tuple t} is a partial mapping that maps attribute names from \textbf{A} to data values in \textbf{D}. Each data-tuple is associated to an event type. We denote $t(a) \in \textbf{D}$ the value of the attribute $a \in \textbf{A}$ assigned by $t$, and $t(type) \in \textbf{T}$ the event type of $t$. If $t$ is not defined on attribute $a$, then we write $t(a) = \text{NULL}$.

A \emph{stream} is a possibly infinite sequence $S = t_{0}t_{1}t_{2}\ldots$ of data-tuples. Given a set $D \subseteq \mathbb{N}$, we define the set of data tuples $S[D] = \{ t_{i} \ | \ i \in D\}$. A \emph{complex event} is a pair $C = ([i,j], D)$ where $i \le j \in \mathbb{N}$ and $D$ is a subset of $\{i, \ldots, j\}$. Intuitively, given a stream $S = t_{0}t_{1}\ldots$ the interval $[i, j]$ of $C$ represents the subsequence $t_{i}t_{i+1} \ldots t_{j}$ of $S$ where the complex event $C$ happens and $S[D]$ represents the data-tuples from $S$ that are relevant for $C$. We write $C(data)$ to denote $D$, $C(time)$ to denote the time-interval $[i, j]$, and $C(start)$ and $C(end)$ for $i$ and $j$, respectively.

Let \textbf{X} be a set of \emph{variables}, which includes all event types, $\textbf{T} \subseteq \textbf{X}$. A \emph{valuation} is a pair $V = ([i, j], \mu)$ with $[i,j]$ a time interval as above and $\mu$ a mapping that assigns subsets of $\{i, \ldots, j\}$ to variables in \textbf{X}. We write $V(time)$, $V(start)$, and $V(end)$ for $[i,j]$, $i$, and $j$, respectively, and $V(X)$ for the subset of $\{i,\ldots, j\}$ assigned to X by $\mu$.

We write $C_{V}$ for the complex event that is obtained from valuation $V$ by forgetting the variables in $V$: $C_{V}(time) = V(time)$ and $C_{V}(data) = \bigcup\limits_{X \in \textbf{X}} V(X)$. The semantics of CEQL will be defined in terms of valuations, which are subsequently transformed into complex events as explained.

\textbf{Predicates}. A (unary) \emph{predicate} is a possibly infinite set $P$ of data-tuples. A data-tuple $t$ \emph{satisfies} predicate $P$, denote $t \Vdash P$, if, and only if, $t \in P$. We generalize this definition from data-tuples to sets by taking a ``for all'' extension: a set of data-tuples $T$ satisfies $P$, denoted by $T \Vdash P$, if, and only if, $\displaystyle\mathop{\forall}_{t \in T} t \vDash P$.
\begin{figure}[H]
  \begin{minted}[xleftmargin=0pt, linenos=false, fontsize=\footnotesize]{text}
    SELECT        [selection strategy] <list of variables>
    FROM          <list of streams>
    WHERE         <CEL formula>
    (PARTITION BY <list of attributes>)?
    (WHITHIN      <time>)?
  \end{minted}
\end{figure}

\vspace{-30pt}
We remark that CEL includes \code{FILTER}, and so CEQL does not need a separate \code{FILTER} clause. The \code{WHERE} clause expects a pattern written in Complex Event Logic (CEL) \cite{formal-framework-cer}, whose abstract syntax is represented by the following grammar:

\vspace{-20pt}
\begin{equation*}
  \varphi := R    \ | \ \varphi \ \text{AS} \ X    \ | \    \varphi \ \text{FILTER} \ X[P]  \ | \   \varphi \ \text{OR} \ \varphi   \ | \  \varphi ; \varphi    \ | \  \varphi+ \ | \ \pi_{L}(\varphi).
\end{equation*}

\vspace{-10pt}
In this grammar, $R$ is an event type in \textbf{T}, $X$ is a variable in \textbf{X}, $P$ is a predicate, and $L$ is a subset of variables in \textbf{X}.

\textbf{Semantics}. The semantics of CEQL is as follows. A CEQL query first evaluates its \code{FROM} clause, then its \code{PARTITION BY} clause, and subsequently its \code{WHERE, SELECT, WITHIN} clauses, in that order. The \code{FROM} clause specifies the list of streams. All these streams are logically merged into a single stream $S$. The optional \code{PARTITION BY} clause logically partitions this stream into multiple substreams $S_{1}, S_{2}\ldots$ and executes the \code{WHERE-SELECT-WITHIN} clauses on each substream. The union of the outputs generated for each substream constitute the final output. Note, the different streams could be evaluated in parallel. CEQL's \code{WHERE} and \code{FILTER} clause are derived from the semantics of CEL in Figure~\ref{fig:cel:semantics}. Specifically, given a stream $S = t_{0}t_{1}t_{2} \ldots $, a CEL formula $\varphi$ evaluates to a set of valuations, denoted $\InSBrackets{\varphi}(S)$. The base case is when $\varphi$ is an event type $R$. In that case $\InSBrackets{\varphi}(S)$ contains all valuations whose time-interval is a single position $i$, such that the data-tuple $t_{i}$ at position $i$ in $S$ is of type $R$. The \code{AS} clause is a variable assignment that takes an existing valuation $V \in \InSBrackets{\varphi}(S)$ and extends it by gathering all positions $\bigcup_{Y}V(Y)$ in variable $X$, keeping all other variables as in $V$. The \code{FILTER} $X[P]$ clause retains only those valuations for which the content of variables $X$ satisfies predicate $P$. The \code{OR} clause takes the union of two sets of valuations. The sequencing operator ($\varphi;\varphi$) uses the time-interval for capturing all pairs of valuations in which the first is chronologically followed by the second. The semantics of iteration ($\varphi+$) is defined as the application of sequencing ($;$) one or more times over the same formula $\varphi$. The projection $\pi_{L}$ modifies valuations by setting all variables that are not in $L$ to empty. Hence, \code{WHERE-FILTER} return a set of valuations when evaluated over a stream. The \code{SELECT} clause, if it does not mention a selection strategy (explained later), corresponds to a projection in CEL. Finally, if $\epsilon$ is a time-interval, then the \code{WITHIN} clause operate on the resulting set of valuations $\InSBrackets{\varphi \ \text{WITHIN} \ \epsilon}(S) = \{ V \in \InSBrackets{S} \ | \ V(end) - V(start) \le \epsilon \}$.

\begin{figure}[t]
  \begin{align*}
    \InSBrackets{R}(S) &= \{V \ | \ V(time) = [i,i] \land t_{i}(type) = R\\
                       &\qquad\quad \land V(R) = {i} \land \forall X \ne R. V(X) = \emptyset \}\\
    \InSBrackets{\varphi \ \text{AS} \ X}(S) &= \{V \ | \ \exists V' \in \InSBrackets{\varphi}(S). V(time) = V'(time)\\
                       &\qquad\quad \land V(X) = \cup_{Y} V'(Y)\\
                       &\qquad\quad \land \forall Z \ne X. V(Z) = V'(Z)\}\\
    \InSBrackets{\varphi \ \text{FILTER} \ X[P]}(S) &= \{V \ | \ V \in \InSBrackets{\varphi}(S) \land V(X) \vDash P \}\\
    \InSBrackets{\varphi_{1} \ \text{OR} \ \varphi_{2}}(S) &= \InSBrackets{\varphi_{1}}(S) \cup \InSBrackets{\varphi_{2}}(S)\\
    \InSBrackets{\varphi_{1} \ ; \ \varphi_{2}}(S) &= \{V \ | \ \exists V_{1} \in \InSBrackets{\varphi_{1}}(S), V_{1} \in \InSBrackets{\varphi_{2}}(S).\\
                       &\qquad\quad  V_{1}(end) = V_{2}(start)\\
                       &\qquad\quad \land V(time) = [V_{1}(start), V_{2}(end)]\\
                       &\qquad\quad \land \forall X. V(X) = V_{1}(X) \cup V_{2}(X) \}\\
    \InSBrackets{\varphi+}(S) &= \InSBrackets{\varphi}(S) \cup \InSBrackets{\varphi;\varphi+}(S)\\
    \InSBrackets{\pi_{L}(\varphi)}(S) &= \{ V \ | \ \exists V' \in \InSBrackets{\varphi}(S). V(time) = V'(time)\\
                              &\qquad\quad \forall X \in L. V(X) = V'(X) \\
                              &\qquad\quad \forall X \notin L. V(X) = \emptyset \}
  \end{align*}
  \caption{The semantics of CEL.}
  \label{fig:cel:semantics}
\end{figure}

\textbf{Complex event semantics}. The complex event semantics of CEL and CEQL is obtained by first evaluating the query under the valuations semantics, and then removing variables. That is, if $\varphi$ is a CEL formula or CEQL query, its complex event semantics $\InDBrackets{\varphi}(S)$ is defined $\InDBrackets{\varphi}(S) := \{ C_{V} \ | \ V \in \InSBrackets{\varphi}(S)\}$.

\section{Selection strategies}\label{sec:selection_strategies}

\emph{Selection strategies} (or \emph{selectors}) are unary operators over CEL formulas that restrict the set of results and speed up query processing. We present four selection strategies \cite{formal-framework-cep,formal-framework-cer}: \emph{strict} (\textsc{strict}), \emph{next} (\textsc{nxt}), \emph{last} (\textsc{last}) and \emph{max} (\textsc{max}). \textsc{strict} and \textsc{nxt} are motivated by the \emph{strict-contiguity} and \emph{skip-till-next-match} selector strategies proposed by SASE \cite{sase}, while \textsc{last} and \textsc{max} are useful selection strategies from a semantic viewpoint \cite{formal-framework-cer}. Next, we describe and formally specify selection strategy \textsc{max}, as it is relevant in our work, and refer the interested reader to \cite{formal-framework-cer} for a definition and discussion of the other selection strategies. For the sake of the discussion, we will define the \emph{support} of a valuation $V$ as the set of all positions appearing in the range of $V$, i.e., $sup(V) = \bigcup\limits_{X \in \textbf{X}}V(X)$.

\textsc{MAX}. This selection strategy keeps the maximal complex events in terms of set inclusion, which could be naturally more useful because these complex events are the \emph{most informative}. Formally, given a CEL formula $\varphi$ we say that $V \in \InSBrackets{\textsc{max}(\varphi)}(S)$ holds iff $V \in \InSBrackets{\varphi}(S)$ and for all $V' \in \InSBrackets{\varphi}(S)$, if $sup(V) \subseteq sup(V')$, then $sup(V) = sup(V')$ (i.e., $V$ is maximal with respect to set containment). For example, given a CEL query $\varphi$, if $\InSBrackets{\varphi}(S)$ returns $\{ 3,6,7\}$, $\{3,4,7\}$, and $\{3,4,6,7\}$. Then, $\textsc{max}(\varphi)$ will only return $\{ 3, 4, 6, 7\}$, which is the maximal complex event.

\section{Computational model}\label{sec:cea}

As explained in Section~\ref{sec:ceql}, evaluating a CEQL query corresponds to evaluating the query's \code{SELECT-WHERE-WITHIN} clauses on either a single stream, or multiple different substreams. In our work, the \code{SELECT-WHERE} part of a query is compiled into a \emph{Complex Event Automaton} \cite{formal-framework-cep,formal-framework-cer}, which is a form of finite automaton that produces complex events. Our evaluation algorithm is then defined in terms of CEA: it takes as input a CEA $\mathcal{A}$, the time-window $\epsilon$ specified in the \code{WITHIN} clause, and a stream $S$, and uses this to compute $\InDBrackets{\varphi}(S)$. Formally, a \emph{Complex Event Automaton (CEA)} is a tuple $\mathcal{A} = (Q, \Delta, q_{0}, F)$ where $Q$ is a finite set of states, $\Delta \subseteq Q \times \textbf{P} \times \{\bullet, \circ\} \times (Q \setminus \{ q_{0} \})$ is a finite transition, $q_{0} \in Q$ is the initial state, and $F \subseteq Q$ is the set of final states. We will denote transitions in $\Delta$ by $q \xrightarrow[]{P/m} q'$. A \emph{run} of $\mathcal{A}$ over stream $S$ from positions $i$ to $j$ is a sequence $\rho := q_{i} \xrightarrow[]{P_{i}/m_{i}} q_{i+1} \xrightarrow[]{P_{i+1}/m_{i+1}} \ldots \xrightarrow[]{P_{j}/m_{j}} q_{j+1}$ such that $q_{i}$ is the initial state of $\mathcal{A}$ and for every $k \in [i,j]$ it holds that $q_{k} \xrightarrow[]{P_{k}/m_{k}} q_{k+1} \in \Delta$ and $t_{k} \vDash P_{k}$. A run $\rho$ is \emph{accepting} if $q_{j+1} \in F$. An accepting run of $\mathcal{A}$ over $S$ from $i$ to $j$ naturally defines the complex event $C_{\rho} := ([i, j], \{ k \ | \ i \le k \le j \land m_{k} = \bullet \})$. Finally, we define the semantics of $\mathcal{A}$ over a stream $S$ as $\InDBrackets{\mathcal{A}}(S) := \{ C_{\rho} \ | \ \rho \text{ is an accepting run of } \mathcal{A} \text{ over } S\}$.

\begin{figure}[t]
  \centering
  \begin{subfigure}[b]{\textwidth}
    \centering
    \inputtikz{cea}
    \vspace*{2em}
  \end{subfigure}
  \begin{subfigure}[t]{\textwidth}
    \centering
    \inputtikz{stream}
  \end{subfigure}

  \caption{A CEA representing the query from Figure~\ref{fig:query:2} and an example of stream.}
  \label{fig:cea}
\end{figure}

\begin{example}
  In Figure~\ref{fig:cea} we show the compilation of the query from Figure~\ref{fig:query:2} into an equivalent CEA $\mathcal{A}$. We depict predicates by listing, in array notation, the event type, and the constraint on the temperature attribute. The initial state is $q_{1}$ and there is only one final state: $q_{4}$. The figure also includes an example stream $S$, where the values correspond to the event type, the identifier attribute, and the temperature attribute, in that order.
\end{example}

The usefulness of CEA comes from the fact that CEL cab be translated into CEA \cite{formal-framework-cep,formal-framework-cer}. Because the \code{SELECT-WHERE} part of a CEQL query is in essence a CEL formula, this reduces the evaluation problem of the \code{SELECT-WHERE-WITHIN} part of CEQL query into the evaluation problem for CEA.

\begin{theorem}[Theorem~1 \cite{core}]\label{theorem:cea}
  For every CEL formula $\varphi$ we can construct a CEA $\mathcal{A}$ of size linear in $\varphi$ such that for every $\epsilon$:
  \begin{equation*}
    \InDBrackets{\varphi \code{ WITHIN } \epsilon}(S) = \{ C \ | \ C \in \InDBrackets{\mathcal{A}}(S) \land C(end) - C(start) \le \epsilon \}
  \end{equation*}
\end{theorem}

Our evaluation algorithm will compute the right-hand side of this equation. It requires that the input CEA $\mathcal{A}$ is \emph{I/O deterministic}: for every pair of transitions $q \xrightarrow[]{P_{1}/m_{1}} q_{1}$ and $q \xrightarrow[]{P_{2}/m_{2}} q_{2}$ from the state $q$, if $P_{1} \cap P_{2} \ne \emptyset$ then $m_{1} \ne m_{2}$. To put it another way, an event $t$ may trigger both transitions at the same time only if one transition marks the event, but the other does not. In \cite{formal-framework-cep,formal-framework-cer}, it was shown that any CEA can be I/O-determinized, with a possibly exponential blow-up in the size of the automaton.

\section{Chapter summary}

In this chapter we have presented the preliminary work that has been published on the related fields of our work. First, we introduced the concept of distributed computing. Then, we described the syntax and semantics of CEQL and CEL. Afterwards, we presented the selection strategies, and explained the semantics of \textsc{max}. Finally, we introduced the computational model CEA, which is used to evaluate CEQL.
