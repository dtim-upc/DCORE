\chapter{???}\label{chapter:???}

%%%%%%%%%%%%%%%%%%%%%%%%%%%%%%%%%%%%%%%%%%%%%%%%%%%%%%%%%%%%

\section{Distribution strategies}\label{sec:distribution_strategies}

TODO

%%%%%%%%%%%%%%%%%%%%%%%%%%%%%%%%%%%%%%%%%%%%%%%%%%%%%%%%%%%%

\section{Distributed enumeration of complex events}
\label{sec:distributed_enumeration_of_the_complex_events}

CORE's evaluation algorithm guarantees, under data complexity, constant time per event and constant-delay enumeration of the output \cite{core}. However, under the default \textit{skip-till-any-match} \cite{skip-till-any-match} policy in CORE, non-contiguous sequencing and iteration can cause the amount of complex events to grow exponentially in the size of the stream \cite{formal-framework-cer}; therefore enumerating all complex events generated by an event is exponential in the worst-case.

% TODO
% Better introduce this
In order to deal with the exponential complexity of enumerating complex events in CER, we propose the distributed enumeration Algorithm \ref{algo:mmde}.

\begin{algorithm}[H]
  \DontPrintSemicolon
  \SetAlgoNoEnd % don't print end
  \SetAlgoNoLine % no vertical lines
  \LinesNumbered
  \SetKwProg{Procedure}{procedure}{}{}
  \SetKwFunction{MMDE}{\textsc{MaximalMatchesDisjointEnumeration}}
  \SetKwFunction{Enumerate}{\textsc{Enumerate}}

  \Procedure{\MMDE{$M$, $W$}}{
    \KwIn{A set of maximal matches $M := \{M_{1}, \ldots, M_{n}\}$ \newline
      and a set of workers $W := \{w_{1},\ldots, w_{m}\}$.
    }
    \KwResult{Enumerates all \textit{submatches} $\subseteq M$ without repetitions.}
    $C \leftarrow \emptyset$\;
    \ForEach{$M_{i} \in M$}{
        $C \leftarrow C \cup \textsc{Configurations}(M_{i}).map(\lambda c \to ( c, M_{i} ))$\;
    }
    $D \leftarrow C.groupBy(\lambda (c, \_ ) \to c)$\;
    $\textsc{Distribute}(W, D)$\tcp*[l]{\textit{Round-robin, Bins \& Balls, etc.}}
    \ForEach{$w \in W$}{
      \tcp{\textsc{Enumerate} is executed in a remote machine}
      $w.\textsc{Enumerate}()$
    }
  }
  \;
\caption{Non-repeated enumeration of a set of maximal matches.}
\label{algo:mmde}
\end{algorithm}

% This procedure enumerates all submatches of M without repetitions.
% It stills enumerates all submatches but only outputs non-repeated.
% It efficiently detects repetitions by constructing an n-ary tree of complex events.
% The complexity is still exponential w.r.t. the size of the largest iteration.
% The exponential time enumeration must be repeated a constant factor of times.
\begin{algorithm}[H]
  \DontPrintSemicolon
  \SetAlgoNoEnd % don't print end
  \SetAlgoNoLine % no vertical lines
  \LinesNumbered
  \SetKwProg{Procedure}{procedure}{}{}
  \SetKwFunction{Enumerate}{\textsc{Enumerate}}
  \SetKwFunction{Enumeratee}{\textsc{Enumerate'}}

  \Procedure{\Enumerate{}}{
    \KwData{A set of tuples $A = \{ (c, \{ M_{1}, \ldots, M_{n}\}) \}$ where $c$ is a \textit{configuration} and $M_{i}$ are maximal matches.}
    \KwOut{The set of all submatches without repetitions.}
    \ForEach{$(c, M) \in A$}{
      $T \leftarrow$ \textit{newRoot()}\tcp*[l]{Root of a \textit{n}-ary tree}
      \ForEach{$M_{i} \in M$}{
        $G \leftarrow \textsc{GroupBy}(M_{i})$\tcp*[l]{$\textsc{GroupBy}(A_{1}A_{2}B_{1}B_{2}C_{1}) = \{ \{A_{1}, A_{2}\}, \{B_{1}, B_{2}\}, \{C_{1}\} \}$}\;
        $\textsc{Enumerate'}(T, G, \emptyset, \bot)$\;
        }
    }
  }
  \;
  \Procedure{\Enumeratee{$n, G, S, new$}}{
    \KwData{A node $n$, a set of grouped events $G$, a time-ordered set of events $S$, and a boolean $new$.}
    \Switch{$G$}{
      \uCase{$\emptyset$}{
        \If{$new$}{
          \Return{$S$}
        }
      }
      \uCase(\tcp*[h]{$e.g. \ g = \{A_{1}, A_{2}\}, \ g.type = A$}){$g \cup G'$}{
        $k \leftarrow c(g.type)$\tcp*[l]{$k \in \mathbb{N}$}
        $E \leftarrow \binom{g}{k}$\tcp*[l]{The \textit{k}-combination set}
        \ForEach{$e \in E$}{
          \eIf{$\exists n' \in n.children \land n'.event = e$}{
            $\textsc{Enumerate'}(n', G', S \cup e, new)$\;
          }{
            $p \leftarrow \text{\textit{new-node}}(e)$\;
            $n.children.add(p)$\;
            $\textsc{Enumerate'}(p, G', S \cup e, \top)$\;
          }
        }
      }
    }
  }
  \;
\caption{Non-repeated enumeration of a set of maximal matches given a predicate configuration.}
\label{algo:enumerate}
\end{algorithm}

\begin{algorithm}[H]
  \DontPrintSemicolon
  \SetAlgoNoEnd % don't print end
  \SetAlgoNoLine % no vertical lines
  \LinesNumbered
  \SetKwProg{Procedure}{procedure}{}{}
  \SetKwFunction{Configurations}{\textsc{Configurations}}

  \Procedure{\Configurations{$M$}}{
    \KwIn{A match $M = \{e_{1}, \ldots, e_{n}\}$ where $e_{i}$ is an event of type $t \in T$.}
    \KwOut{A set $C$ of configurations $c := T \times \mathbb{N}$ where $c$ is the mapping from the event type $t \in T$ to the size of the iteration of the event type $t$ in the submatches of $M$.}
    $V \leftarrow newList$\;
    $e_{0} \cup M' \leftarrow pop(M)$\;
    $A \leftarrow \{ e_{0} \}$\;
    $A.type \leftarrow e_{0}.type$\;
    \For{event $e$ in $M'$}{
      \eIf{$e.type = A.type$}{
        $A \leftarrow A \cup e$\;
        \uIf{$isLast(e)$} {
          $V \leftarrow V + enumFromTo(1, |A|)$
        }
      }{
        $V \leftarrow V + enumFromTo(1, |A|)$\;
        $A \leftarrow \{ e \}$\;
        $A.type \leftarrow e.type$\;
      }
    }
    $WW \leftarrow V_{1} \times \cdots \times V_{n}$\tcp*[l]{$V = \{V_{1}, \cdots, V_{n}\}$}
    $T \leftarrow types(M)$\tcp*[l]{Ordered set of types e.g. $types(A_{1}A_{2}B_{1}C_{1}) = \{A,B,C\}$}
    $C \leftarrow \emptyset$\;
    \ForEach(\tcp*[h]{E.g. $W = \{1, 2, 1\}$}){$W \in WW$}{
      $c \leftarrow \emptyset$\tcp*[l]{E.g. $c = \{(A,1), (B,2), (C, 1)\}$}
      \For{$i \leftarrow 1$ \KwTo $|W|$}{
        $c \leftarrow c \cup (T[i], W[i])$\;
      }
      $C \leftarrow C \cup c$\;
    }
    \Return{C}
  }
\caption{Computes all disjoint configurations of a maximal match.}
\label{algo:configurations}
\end{algorithm}

% You need to make the following observations of "Maximal Matches Enumeration":
% 1. The algorithm produces disjoint submatches given a maximal match.
% 2. The algorithm produces non-disjoint submatches given multiple maximal matches.

% But (2) can be analyzed further:
% 1. Disjoint configurations produce disjoint submatches.
% 2. Non-disjoint configurations produce non-disjoint submatches.

% From previous observations we can conclude that repeated submatches are only generated by applying the same configuration to different maximal matches.

% Uniqueness of submatches is guaranteed by (3) and (4).
% (3) guarantees that the output of each worker is disjoint wrt the others.
% (4) guarantees that the output of a worker is disjoint.

% The complexity of the algorithm remains the same if we accomplish linear time enumeration in each worker (this is the tricky part).

\begin{lemma}[TODO]
  \label{lemma:todo}
  TODO
\end{lemma}

\begin{theorem}[TODO]
  \label{theorem:todo}
  TODO
\end{theorem}

\begin{example}[TODO]
  \label{example:todo}
  TODO
\end{example}

% Soundness and Completness of an Algorithm
%
% Let S be the set of all right answers.
% A sound algorithm never includes a wrong answer in S, but it might miss a few right answers.
% A complete algorithm should get every right answer in S: include the complete set of right answers. But it might include a few wrong answers.
%
% Careful! Soudness and completness in logic has another meaning https://math.stackexchange.com/questions/105575/what-is-the-difference-between-completeness-and-soundness-in-first-order-logic

%%%%%%%%%%%%%%%%%%%%%%%%%%%%%%%%%%%%%%%%%%%%%%%%%%%%%%%%%%%%

\newpage
\section{Distributed tECS enumeration}\label{sec:distributed_tecs_enumeration}

CORE evaluation algorithm:

\begin{itemize}
  \item Constant time to update the data structure with a new input event, and the size of the data structure is at most linear in the number of seen events.
  \item The enumeration is done in output-linear delay without duplicates, meaning that the time required to output recognized complex event $C$ is linear in the size of $C$.
\end{itemize}

Problem:

\begin{itemize}
  \item The number of complex events generated by an event is at most exponential in the length of the stream.
\end{itemize}

For example, a query like

\begin{equation}
  \varphi_{1}= \text{(T; H+ as HS; H as LH) FILTER (T.temp } < 40 \ \land \ \text{HS.hum} < 60 \ \land \ \text{LH.hum} > 60)
  \label{eq:phi_1}
\end{equation}

will generate exponential number of complex events in the length of the kleene closure of \ref{eq:phi_1} \cite{on-the-expressiveness}

Our goal is to design an algorithm that delegates the enumeration of the complex events to $N$ distributed machines in a cluster while preserving the efficiency of CORE's evaluation algorithm.

% In order to guarantee constant update and output-linear delay, the algorithm will extend Algorithm 1 and 2 from CORE[1].

% Algorithm 1:

% (1.1) We need to modify new-bottom, extend and union such that each methods adds to the nodes the number of paths up to the node:
% - new-bottom adds $paths(b) = 1$
% - extend adds $paths(o) = paths(n)$
% - (Informally, since we need to take into account the 4 cases) union-node adds to each union node the paths information to left, right and itself (left + right)

% The $paths$ information will be used later for the enumeration process on each worker.

% (1.2) We also need to distributed this extended-tECS among the N workers **in constant time per event**. Consequently, we need to incrementally transfer the data structure to the workers. A naive solution would be to replicate the evaluation Algorithm 1 in each worker and only distribute the events. A better solution would be to incrementally generate the extended-tECS at the workers by sending, on each modification of the tECS, a command of constant size to each worker with the instructions to recreate the extended-tECS without having to evaluate the CEA nor the hash table. It is obvious that each worker will need space equal to the length of the stream.

% (1.1) and (1.2) can be done in constant update time.

% Algorithm 2:

% ENUMERATE will be executed in each worker after an event on a final state is received (this can be marked on the instruction send with the event). ENUMERATE must be executed as efficiently as possible i.e. preserving linear-output delay and minimum communication between workers. We emphasize that the proposed algorithm does not require communication between the workers by determinizing the assigned workload and the walk-through of the tECS.

% Notice, this algorithm is executed in parallel in each worker independently of the rest of the workers.

% (2.1) First, we need to add between line 3 and 4:
% - a = floor (paths(n)/N) where N is the number of workers (this is a constant)
% - index(worker); each worker has assigned an index in increasing order starting from 0.
% - s = index*a; s marks the number of paths enumerate so far by previous workers. For example, worker_0 will have s = 0.
% - c = a; c is the number of paths that this worker needs to enumerate.

% (2.2) We need to modify lines 14-17 from Algorithm 2 in CORE[1]. We present the recursive implementation and leave for later the translation to the imperative version which should be trivially adapted.

% ENUMERATE_REC(n, s, c):
%     If paths(right(n)) > s // Not all right paths have been traversed by previous workers
%         ENUMERATE_REC(right(n), s, c)
%         c -= max(0, paths(right(n)) - s)
%     s = max(0, s - paths(right(n)))
%     If c > 0 and paths(left(n)) > s // There is work left to do and left still has paths not traversed by previous workers.
%         ENUMERATE_REC(left(n), s, c)

% (2.2) can be computed in constant time since we only added an extra conditional to the left traversal of the tree and paths can be access in constant time.




\begin{algorithm}[H]
  \DontPrintSemicolon
  \SetAlgoNoEnd % don't print end
  \SetAlgoNoLine % no vertical lines
  \LinesNumbered
  \SetKwProg{Procedure}{procedure}{}{}
  \SetKwFunction{Distribute}{\textsc{Distribute}}
  \SetKwFunction{Enumerate}{\textsc{Enumerate}}

  \Procedure{\Distribute{$\mathcal{E}, n, \epsilon, j, P, W$}}{
     \nl $b <- bitstring(|P|)$\;
     \nl $B <- permutations(b)$\;
     $\mathcal{G} \leftarrow \emptyset$\;
     \ForEach{$b \in B$}{
       $G = \emptyset$\tcp*[l]{$f: N_{U} \rightarrow \{ \bot , \top \}$}
       \For{$i \leftarrow 1$ \KwTo $|p|$}{
           $G \leftarrow G \cup (P_{i}, b_{i})$\;\label{algo:de:7}
       }
       $\mathcal{G} \leftarrow \mathcal{G} \cup G$\;
     }
     $S \leftarrow \emptyset$\tcp*[l]{Assignments}\label{algo:de:9}
     \For{$i \leftarrow 1$ \KwTo $|\mathcal{G}|$}{
       \nllabel{2}$i' \leftarrow i \mod |W|$\;
       $S \leftarrow S \cup (S_{i'} \cup \{ \mathcal{G}_{i} \} )$\;
     }\label{algo:de:12}
     \For{$i \leftarrow 1$ \KwTo $|W|$}{
        $W_{i}.\textsc{Enumerate}(\mathcal{E}, n, \epsilon, j, S_{i})$\;
     }
  }
\caption{Distributed enumeration of $\llbracket n \rrbracket^{\epsilon}_{\mathcal{E}}(j)$}
\label{algo:de}
\end{algorithm}

\setlength{\interspacetitleruled}{-.4pt}%
\begin{algorithm}[H]
  \DontPrintSemicolon
  \SetAlgoNoEnd % don't print end
  \SetAlgoNoLine % no vertical lines
  \LinesNotNumbered
  \SetKwProg{Procedure}{procedure}{}{}
  \SetKwFunction{Enumerate}{\textsc{Enumerate}}
  \nl \Procedure{\Enumerate{$\mathcal{E}, n, \epsilon, j, \mathcal{G}$}}{
    $\cdots$\;
    \setcounter{AlgoLine}{13}
    \nl \ElseIf{$n' \in N_{U}$}{
      \nl$choice \leftarrow (\bot, \bot)$\tcp*[l]{(left, right)}
      \nl\ForEach{$G \in \mathcal{G}$}{\label{algo:enumerate:16}
        \nl\If{$G_{1}(u)$}{
          \nl$choice.right = \top$\;
        }
        \nl\Else{
          \nl$choice.left = \top$\;
        }
        \nl$G \leftarrow G \setminus \{ G_{i} \}$\;\label{algo:enumerate:21}
      }
      \nl\If{$choice.right \land max(right(n')) \geq \tau$}{\label{algo:enumerate:22}
        \nl$push(st, (right(n'), P))$\;
      }
      \nl\ElseIf{$choice.left$}{\label{algo:enumerate:24}
        \nl$n' \leftarrow left(n')$\;
      }
    }
    $\cdots$\;
  }
\label{algo:enumerate}
\end{algorithm}

We present Algorithm \ref{algo:de} which distributes the workload of \textsc{Enumerate} (Algorithm 1) from \cite{core} among $N$ workers. In order to incrementally distribute the enumeration of $\mathcal{E}$, we need to manipulate an additional ordered set $P$ of union nodes during the evaluation of \textsc{Evaluation}. This set will store all unions nodes present in $\mathcal{E}$ during the evaluation of the \gls{cea}. The algorithm consist of two distinct procedures: \textsc{Distribute} and \textsc{Enumerate}. \textsc{Distribute} replaces \textsc{Enumerate} in the original algorithm and \textsc{Enumerate} is executed remotely on each worker.

In detail, Algorithm \ref{algo:de} receives as an input a gls{tecs} $\mathcal{E}$, a node $n$, a time-window bound $\epsilon$, a position $j$, an ordered set of union nodes $P$ and, a set of workers $W$. First, creates a bitstring (e.g. $111111$) of size $|p|$ where each position corresponds to a union node in $\mathcal{E}$ and computes the $2^{n}$ permutations of this bitstring i.e. all possible paths that traverses the \gls{dag} $G_{\mathcal{E}}$ encoded by $\mathcal{E}$. Then, for each possible path, line \ref{algo:de:7} computes the mapping between the union node $P_{i}$ and the choice $b_{i}$ ($\top \equiv \text{ right}$ and $\bot \equiv \text{ left}$). Then, lines \ref{algo:de:9}-\ref{algo:de:12} distribute the paths among all workers. We chose the simples load-balancing algorithm, the round-robin but these lines can be replaced by any load-balancing algorithm. Finally, we call \textsc{Enumerate} on each worker using a \gls{rmi}.

On the other side of the cluster, each worker will be waiting for the remote call \textsc{Enumerate}. \textsc{Enumerate} is a modification of the algorithm \textsc{Enumerate} (Algorithm 2) from \cite{core}. The procedure receives a set $\mathcal{G}$ of binary relations $G$ that are the choices to be made on each union node. Notice, the traversing of the \gls{dag} can be done in a single traversal if we combine all the path choices at each union node (lines \ref{algo:enumerate:16}-\ref{algo:enumerate:21}). Then lines \ref{algo:enumerate:22} and \ref{algo:enumerate:24} have been modified to accommodate this choice.

(we need to prove this, it should be easy) Algorithm \ref{algo:de} still enumerates the set ${\llbracket n \rrbracket}^{\epsilon}_{\mathcal{E}}(j)$ in output-liner delay and without repetitions.

% CEA AB+
\begin{figure}[H]
  \centering
  \begin{subfigure}[t]{\textwidth}
    \centering
    \inputtikz{streamAB+}
  \end{subfigure}
  \\
  \begin{subfigure}[b]{\textwidth}
    \begin{minted}[xleftmargin=40pt, linenos=false]{text}
      SELECT *
      FROM S
      WHERE A as a; B + as bb
    \end{minted}
  \end{subfigure}
  \\
  \begin{subfigure}[b]{\textwidth}
    \centering
    \inputtikz{ceaAB+}
  \end{subfigure}
  \caption{A CEA representing $Q_{1}$ from Figure 1 and some of its runs on an example stream.}
  \label{fig:label}
\end{figure}

\begin{figure}[H]
  \centering
  \begin{subfigure}[t]{0.1\linewidth}
    \inputtikz{AB+_0}
  \end{subfigure}
  \begin{subfigure}[t]{0.1\linewidth}
    \inputtikz{AB+_1}
  \end{subfigure}
  \begin{subfigure}[t]{0.24\linewidth}
    \inputtikz{AB+_2}
  \end{subfigure}
  \begin{subfigure}[t]{0.24\linewidth}
    \inputtikz{AB+_3}
  \end{subfigure}
  \begin{subfigure}[t]{0.28\linewidth}
    \inputtikz{AB+_4}
  \end{subfigure}
  \caption{Illustration of Algorithm 1 on the CEA $\mathcal{A}$ and stream $S$ of Figure ???.}
  \label{fig:label}
\end{figure}

% CEA A+B+
\begin{figure}[H]
  \centering
  \begin{subfigure}[t]{\textwidth}
    \centering
    \inputtikz{streamA+B+}
  \end{subfigure}
  \\
  \begin{subfigure}[b]{\textwidth}
    \begin{minted}[xleftmargin=40pt, linenos=false]{text}
      SELECT *
      FROM S
      WHERE A + as aa; B + as bb
    \end{minted}
  \end{subfigure}
  \\
  \begin{subfigure}[b]{\textwidth}
    \centering
    \inputtikz{ceaA+B+}
  \end{subfigure}
  \caption{A CEA representing $Q_{1}$ from Figure 1 and some of its runs on an example stream.}
  \label{fig:label}
\end{figure}

\begin{figure}[H]
  \begin{subfigure}[t]{0.1\linewidth}
    \inputtikz{A+B+_0}
  \end{subfigure}
  \begin{subfigure}[t]{0.1\linewidth}
    \inputtikz{A+B+_1}
  \end{subfigure}
  \begin{subfigure}[t]{0.24\linewidth}
    \inputtikz{A+B+_2}
  \end{subfigure}
  \begin{subfigure}[t]{0.24\linewidth}
    \inputtikz{A+B+_3}
  \end{subfigure}
  \begin{subfigure}[t]{0.28\linewidth}
    \inputtikz{A+B+_4}
  \end{subfigure}
  \caption{Illustration of Algorithm 1 on the CEA $\mathcal{A}$ and stream $S$ of Figure ???.}
  \label{fig:label}
\end{figure}

%%%%%%%%%%%%%%%%%%%%%%%%%%%%%%%%%%%%%%%%%%%%%%%%%%%%%%%%%%%%

\section{Chapter summary}

TODO
