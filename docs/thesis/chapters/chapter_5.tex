\chapter{Distributed CER}\label{chapter:distributed-cer}

In this chapter we propose a novel framework for distributed CER that is based on CEL and CEA from Chapter~\ref{chapter:preliminaries}. We describe a first implementation based on this framework. Furthermore, we introduce several distributions strategies to be used on this implementation. Finally, we introduce an alternative implementation based on the distributed evaluation algorithm from Chapter~\ref{chapter:algorithm}. We show in Chapter~\ref{chapter:experimental_evaluation} that the latter implementation performance is overall better thanks to its ad-hoc evaluation algorithm.

\section{Proposed approach}\label{sec:proposed-approach}

Given the efficient evaluation of CEA's algorithms in a centralized manner (see \cite{formal-framework-cer, core}), it is hard to foresee scenarios where it can benefit from distribution. Indeed, both automata distribution approaches (i.e., query partitioning and pipelining) will incur a big overhead in terms of network communication during the evaluation process to synchronize the tECS $\mathcal{E}$ and the enumeration process. However, as previously discussed, CEA is very limited in terms of filtering capabilities, allowing only \emph{unary} predicates. Thus, one might wonder how to include the evaluation of more complex filters over non-unary predicates such as binary predicates (e.g., an equi-join \code{h1[id] = t2[id]}), or second-order predicates (e.g., the sequence of \code{T[val]} must monotonically increase). In \cite{on-the-expressiveness}, it is discussed that CEL and CEA are equivalent in expressive power when CEL is restricted to unary predicates, but incomparable in general. Thus, non-unary CEL, in general, cannot be compiled into an equivalent CEA. However, one could split the pattern matching process and the filtering in CER. In other words, we would maintain the generation of complex events in CEA, but leverage on a distributed framework for complex filtering.



















\section{Prototype I}\label{sec:prototype:I}

% \item A distributed system for evaluating non-unary predicates over CEQL queries. The CEQL query is feed into the system and processed by a single actor. This actor parses the query, analyses its semantics, and generates the corresponding unary CEQL, and an evaluation plan for the non-unary predicates. The unary CEQL is processed sequentially using the efficient evaluation process of \cite{formal-framework-cep} and complex events are output. Then, a \emph{distribution strategy} is used to distribute the complex events, with additional non-unary predicate information, between the nodes of the distributed system. Finally, each node of the system, evaluates the non-unary predicates over the resulting complex events.

\subsection{Distribution Strategies}\label{subsec:distribution-strategies}

We propose a novel enumeration algorithm called \acrfull{mmde}

\begin{algorithm}[H]
  \DontPrintSemicolon
  \SetAlgoNoEnd % don't print end
  \SetAlgoNoLine % no vertical lines
  \LinesNumbered
  \SetKwProg{Procedure}{procedure}{}{}
  \SetKwFunction{MMDE}{\textsc{MaximalMatchesDisjointEnumeration}}
  \SetKwFunction{Enumerate}{\textsc{Enumerate}}

  \Procedure{\MMDE{$M$, $W$}}{
    \KwIn{A set of maximal matches $M := \{M_{1}, \ldots, M_{n}\}$ \newline
      and a set of workers $W := \{w_{1},\ldots, w_{m}\}$.
    }
    \KwResult{Enumerates all \emph{submatches} $\subseteq M$ without repetitions.}
    $C \leftarrow \emptyset$\;
    \ForEach{$M_{i} \in M$}{
        $C \leftarrow C \cup \textsc{Configurations}(M_{i}).map(\lambda c \to ( c, M_{i} ))$\;
    }
    $D \leftarrow C.groupBy(\lambda (c, \_ ) \to c)$\;
    $\textsc{Distribute}(W, D)$
  }
  \;
\caption{Non-repeated enumeration of a set of maximal matches.}
\label{algo:mmde}
\end{algorithm}

% This procedure enumerates all submatches of M without repetitions.
% It stills enumerates all submatches but only outputs non-repeated.
% It efficiently detects repetitions by constructing an n-ary tree of complex events.
% The complexity is still exponential w.r.t. the size of the largest iteration.
% The exponential time enumeration must be repeated a constant factor of times.
\begin{algorithm}[H]
  \DontPrintSemicolon
  \SetAlgoNoEnd % don't print end
  \SetAlgoNoLine % no vertical lines
  \LinesNumbered
  \SetKwProg{Procedure}{procedure}{}{}
  \SetKwFunction{Enumerate}{\textsc{Enumerate}}
  \SetKwFunction{Enumeratee}{\textsc{Enumerate'}}

  \Procedure{\Enumerate{}}{
    \KwData{A set of tuples $A = \{ (c, \{ M_{1}, \ldots, M_{n}\}) \}$ where $c$ is a \emph{configuration} and $M_{i}$ are maximal matches.}
    \KwOut{The set of all submatches without repetitions.}
    \ForEach{$(c, M) \in A$}{
      $T \leftarrow$ \text{new-root()}
      \ForEach{$M_{i} \in M$}{
        $G \leftarrow \textsc{GroupBy}(M_{i})$\;
        $\textsc{Enumerate'}(T, G, \emptyset, \bot)$\;
        }
    }
  }
  \;
  \Procedure{\Enumeratee{$n, G, S, new$}}{
    \KwData{A node $n$, a set of grouped events $G$, a time-ordered set of events $S$, and a boolean $new$.}
    \Switch{$G$}{
      \uCase{$\emptyset$}{
        \If{$new$}{
          \Return{$S$}
        }
      }
      \uCase{$g \cup G'$}{
        $k \leftarrow c(g.type)$
        $E \leftarrow \binom{g}{k}$
        \ForEach{$e \in E$}{
          \eIf{$\exists n' \in n.children \land n'.event = e$}{
            $\textsc{Enumerate'}(n', G', S \cup e, new)$\;
          }{
            $p \leftarrow$ new-node($e$)\;
            $n.children.add(p)$\;
            \textsc{Enumerate'}$(p, G', S \cup e, \top)$\;
          }
        }
      }
    }
  }
  \;
\caption{Non-repeated enumeration of a set of maximal matches given a predicate configuration.}
\label{algo:enumerate}
\end{algorithm}

\begin{algorithm}[H]
  \DontPrintSemicolon
  \SetAlgoNoEnd % don't print end
  \SetAlgoNoLine % no vertical lines
  \LinesNumbered
  \SetKwProg{Procedure}{procedure}{}{}
  \SetKwFunction{Configurations}{\textsc{Configurations}}

  \Procedure{\Configurations{$M$}}{
    \KwIn{A match $M = \{e_{1}, \ldots, e_{n}\}$ where $e_{i}$ is an event of type $t \in T$.}
    \KwOut{A set $C$ of configurations $c := T \times \mathbb{N}$ where $c$ is the mapping from the event type $t \in T$ to the size of the iteration of the event type $t$ in the submatches of $M$.}
    $V \leftarrow newList$\;
    $e_{0} \cup M' \leftarrow pop(M)$\;
    $A \leftarrow \{ e_{0} \}$\;
    $A.type \leftarrow e_{0}.type$\;
    \For{event $e$ in $M'$}{
      \eIf{$e.type = A.type$}{
        $A \leftarrow A \cup e$\;
        \uIf{$isLast(e)$} {
          $V \leftarrow V + enumFromTo(1, |A|)$
        }
      }{
        $V \leftarrow V + enumFromTo(1, |A|)$\;
        $A \leftarrow \{ e \}$\;
        $A.type \leftarrow e.type$\;
      }
    }
    $WW \leftarrow V_{1} \times \cdots \times V_{n}$\tcp*[l]{$V = \{V_{1}, \cdots, V_{n}\}$}
    $T \leftarrow types(M)$\tcp*[l]{Ordered set of types e.g. $types(A_{1}A_{2}B_{1}C_{1}) = \{A,B,C\}$}
    $C \leftarrow \emptyset$\;
    \ForEach(\tcp*[h]{E.g. $W = \{1, 2, 1\}$}){$W \in WW$}{
      $c \leftarrow \emptyset$\tcp*[l]{E.g. $c = \{(A,1), (B,2), (C, 1)\}$}
      \For{$i \leftarrow 1$ \KwTo $|W|$}{
        $c \leftarrow c \cup (T[i], W[i])$\;
      }
      $C \leftarrow C \cup c$\;
    }
    \Return{C}
  }
\caption{Computes all disjoint configurations of a maximal match.}
\label{algo:configurations}
\end{algorithm}

% You need to make the following observations of "Maximal Matches Enumeration":
% 1. The algorithm produces disjoint submatches given a maximal match.
% 2. The algorithm produces non-disjoint submatches given multiple maximal matches.

% But (2) can be analyzed further:
% 1. Disjoint configurations produce disjoint submatches.
% 2. Non-disjoint configurations produce non-disjoint submatches.

% From previous observations we can conclude that repeated submatches are only generated by applying the same configuration to different maximal matches.

% Uniqueness of submatches is guaranteed by (3) and (4).
% (3) guarantees that the output of each worker is disjoint wrt the others.
% (4) guarantees that the output of a worker is disjoint.

% The complexity of the algorithm remains the same if we accomplish linear time enumeration in each worker (this is the tricky part).


\section{Prototype II}\label{sec:prototype:II}

% \item Another distributed system for evaluating non-unary predicates over CEQL queries based on the distributed evaluation algorithm presented in Contributions~(I). The system differs from the previous in that: the query is feed to \emph{all} the nodes, each node parses and extracts the corresponding unary CEQL query and a plan for evaluating the non-unary predicates of the query, and the evaluation algorithm is executed on each node. During the distributed enumeration phase of the evaluation algorithm, the output is filtered taking into account the non-unary predicates.


\section{Chapter summary}

