\chapter{Preliminaries}\label{chapter:preliminaries}

% TODO move this to the first time constant-delay is mentioned.
The notion of constant-delay enumeration was defined in the database community \cite{Segoufin13enumeratingwith, 10.1007/978-3-540-74915-8_18} for defining efficiency whenever generating the output might use considerable time. An enumeration is performed in constant-delay if it takes constant time between any two consecutive solutions.

% Same for data complexity
There are three ways to measure the complexity of evaluating queries in a specific language. First, one can fix a specific query in the language and study the complexity of applying this query to arbitrary databases. The complexity is then given as a function of the size of the databases. We call this complexity \textit{data complexity} \cite{DataComplexity}.

\section{CORE}

\begin{definition}[Match]
  \label{def:match}
  TODO
\end{definition}

\begin{definition}[Maximal Match]
  \label{def:maximalmatch}
  TODO
\end{definition}

\section{SO-CEL}

\section{Chapter summary}
