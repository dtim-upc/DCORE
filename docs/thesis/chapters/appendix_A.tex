\chapter{}\label{appendix:A}

\section{Proof of Theorem~\ref{theorem:isomorphism}}\label{appendix:A:sec:1}

\begin{theorem}[${\llbracket \text{n} \rrbracket}^{\epsilon}_{\mathcal{E}}(j) \longleftrightarrow paths_{\ge j - \epsilon}(\text{n})$]
  For every complex event within a time window of size $\epsilon$ there exists exactly one path that reaches a bottom node $b$ with $pos(b) \ge j - \epsilon$, and vice versa.
\end{theorem}

\begin{proof}
  Fix $j$, $\epsilon$, and $\mathcal{E}$. Let \textrm{n} be a node in $\mathcal{E}$. The proof follows by the definition of ${\llbracket \text{n} \rrbracket}^{\epsilon}_{\mathcal{E}}(j)$, ${\llbracket \text{n} \rrbracket}_{\mathcal{E}}$, ${\llbracket \bar{p} \rrbracket}_{\mathcal{E}}$, and ${paths}_{\ge \tau}(\text{n})$. Recall that

  \begin{itemize}
    \item ${\llbracket \text{n} \rrbracket}^{\epsilon}_{\mathcal{E}}(j) = \{ ([i,j], D) \ | \ (i,D) \in {\llbracket \text{n} \rrbracket}_{\mathcal{E}} \land j - i \le \epsilon \}$ encodes all open complex events represented by \textrm{n} in $\mathcal{E}$ that, when closed with j, are within a time window of size $\epsilon$,
    \item ${\llbracket \text{n} \rrbracket}_{\mathcal{E}} = \bigcup\limits_{\bar{p};\\ start(\bar{p}) = n} {\llbracket \bar{p} \rrbracket}_{\mathcal{E}}$ encodes all open complex events ${\llbracket \bar{p} \rrbracket}_{\mathcal{E}}$ with $\bar{p}$ a full-path in $\mathcal{E}$ starting at \textrm{n}, and
    \item ${\llbracket \bar{p} \rrbracket}_{\mathcal{E}} = (i, D)$ where $\bar{p} = n_{1},n_{2}, \ldots, n_{k}$ be a \emph{full-path} in $\mathcal{E}$ such that $n_{k}$ is a bottom node, $i = pos(n_{k})$ is the label of the bottom node $n_{k}$, and $D$ is the set of labels of the other non-union nodes in $\bar{p}$.
  \end{itemize}

  First, we prove ${\llbracket \text{n} \rrbracket}^{\epsilon}_{\mathcal{E}}(j) \longmapsto paths_{\ge j - \epsilon}(\text{n})$. Given a complex event $([i, j], D) \in {\llbracket \text{n} \rrbracket}^{\epsilon}_{\mathcal{E}}(j)$, there is an open complex event $(i, D) \in {\llbracket \text{n} \rrbracket}_{\mathcal{E}}$ that is represented as the full-path $\bar{p} = n_{1},n_{2}, \ldots, n_{k}$ in $\mathcal{E}$ such that $n_{k}$ is a bottom node and $i = pos(n_{k})$ is the label of the bottom node $n_{k}$. Notice, $n_{1} = \textrm{n}$ is the starting node, $j = pos(n_{1})$ is the label of the starting node $n_{1}$, and $j - i \le \epsilon$. By definition, $\bar{p} \in {paths}_{\ge \tau}(\text{n})$.

  Secondly, we prove that $paths_{\ge j - \epsilon}(\text{n}) \longmapsto {\llbracket \text{n} \rrbracket}^{\epsilon}_{\mathcal{E}}(j)$. The proof follows by expanding the definition of ${paths}_{\ge \tau}(\text{n})$ and following the steps of the previous proof in reverse order.

  Finally, by ${\llbracket \text{n} \rrbracket}^{\epsilon}_{\mathcal{E}}(j) \longmapsto paths_{\ge j - \epsilon}(\text{n})$ and $paths_{\ge j - \epsilon}(\text{n}) \longmapsto {\llbracket \text{n} \rrbracket}^{\epsilon}_{\mathcal{E}}(j)$, ${\llbracket \text{n} \rrbracket}^{\epsilon}_{\mathcal{E}}(j) \longleftrightarrow paths_{\ge j - \epsilon}(\text{n})$ immediately holds.

\end{proof}

\section{Algorithms Chapter~\ref{chapter:distributed-cer}}\label{appendix:A:sec:2}

\subsection{Maximal Complex Event Enumeration}\label{appendix:A:sec:2:subsec:1}

\begin{algorithm}[H]
  \setstretch{0.9} % space between lines
  \DontPrintSemicolon
  \SetAlgoNoEnd % don't print end
  \SetAlgoNoLine % no vertical lines
  \LinesNumbered
  \SetKwProg{Procedure}{procedure}{}{}
  \SetKwFunction{MCEE}{\textsc{Mcee}}
  \SetKwFunction{Enumerate}{\textsc{Enumerate}}
  \SetKwFunction{Loop}{\textsc{Loop}}
  \SetKwFunction{Configurations}{\textsc{Configurations}}
  \SetKwFunction{Distribute}{\textsc{Distribute}}
  \SetKwFunction{Partition}{\textsc{Partition}}
  \SetKwFunction{Head}{head}
  \SetKwFunction{Type}{type}
  \SetKwFunction{Last}{last}
  \SetKwFunction{Enum}{enum}
  \SetKwFunction{GroupBy}{\textsc{GroupBy}}
  \SetKwFunction{Type}{type}
  \SetKwFunction{Event}{event}
  \SetKwFunction{Children}{children}
  \SetKwFunction{Add}{add}
  \SetKwFunction{Map}{\textsc{Map}}
  \SetKwFunction{NewNode}{new-node}

  \Procedure{\MCEE{$\mathcal{C}^{\text{max}}_{V}$, $\mathcal{P}$}}{
    $\mathcal{K} \leftarrow \emptyset$\;
    \For{$C^{\text{max}}_{V} \in \mathcal{C}^{\text{max}}_{V}$}{\label{algo:mcee:line:for:1}
        $\mathcal{K} \leftarrow \mathcal{K} \cup \Configurations{$C_{V}^{\text{max}}$}.\Map{$\lambda K \to \langle K, C_{V}^{\text{max}} \rangle$}$\;
    }
    $D \leftarrow $\GroupBy{$\mathcal{K}, \lambda \langle K, \_ \rangle \to K$}\;\label{algo:mcee:line:groupby}
    \Distribute{$\mathcal{P}, D$}\;
  }
  \;

  \Procedure{\Enumerate{$D$}}{
    \For{$\langle K, \mathcal{C^{\text{max}}_{V}} \rangle \in D$}{\label{algo:mcee:line:for:2}
      $T \leftarrow$ \text{\NewNode{}}\;\label{algo:mcee:line:new-node}
      \For{$C^{\text{max}}_{V} \in \mathcal{C}^{\text{max}}_{V}$}{\label{algo:mcee:line:for:3}
        $\mathcal{G} \leftarrow \Partition{$C_V^{\text{max}}$}$\;
        \Loop{$T, \mathcal{G}, \emptyset, \bot, K$}\;
        }
    }
  }
  \;

  \Procedure{\Loop{$n, \mathcal{G}, C_{V}, new, K$}}{
    \Switch{$\mathcal{G}$}{\label{algo:mcee:line:switch}
      \uCase{$\emptyset$}{\label{algo:mcee:line:case:1}
        \If{$new$}{\label{algo:mcee:line:if:1}
          \Return{$C_{V}$}
        }
      }
      \uCase{$G \cup \mathcal{G}$}{\label{algo:mcee:line:case:2}
        $k \leftarrow K(\Type{G})$\;
        $N \leftarrow \binom{G}{k}$\;
        \For{$i \in N$}{
          \eIf{$\exists n' \in \Children{n}. \ \ \Event{n'} = i$}{\label{algo:mcee:line:if:2}
            \Loop{$n', \mathcal{G}, C_{V} \cup i, new$}\;
          }{
            $n' \leftarrow $\NewNode{$i$}\;
            $\Add{\Children{n}, n'}$\;
            \Loop{$n', \mathcal{G}, C_{V} \cup i, \top, K$}\;
          }
        }
      }
    }
  }
\caption{Distributed enumeration of a set of maximal complex events $\mathcal{C}^{\text{max}}_{V}$ over a set of processing units $\mathcal{P}$.}
\label{algo:mcee}
\end{algorithm}

Algorithm~\ref{algo:mcee} consist of two procedures: \textsc{Mcee} and \textsc{Enumerate}. The main procedure is \textsc{Mcee}, which is executed in the master actor, while \textsc{Enumerate} is executed on each slave actor. It receives as an input a set of \emph{maximal} complex events $\mathcal{C}^{\text{max}}_{V} := \{C^{1}_{V}, \ldots, C^{n}_{V}\}$ and a set of processing unit $\mathcal{P} := \{P_{1},\ldots, P_{n}\}$, and outputs all \emph{complex events} $C_{V}' \subseteq \mathcal{C}^{\text{max}}_{V}$ distributedly. The \code{for} (lines~3-4) compute the set of \emph{configurations} corresponding to each maximal complex event in $\mathcal{C}^{\text{max}}_{V}$ (see Algorithm~\ref{algo:configurations}) and pairs each configuration with its maximal complex event. A \emph{configuration} is a binary relation $\textbf{T} \times \mathbb{N}$ from data-tuples $t \in \textbf{T}$ to natural numbers. For example, given complex event $C_{V} := \{T, T, H, H, H\}$, then $\code{Configurations}(C_{V}) = \{ \{T, 2\}, \{H, 3\}\}$.


\begin{algorithm}[H]
  \setstretch{0.9} % space between lines
  \DontPrintSemicolon
  \SetAlgoNoEnd % don't print end
  \SetAlgoNoLine % no vertical lines
  \LinesNumbered
  \SetKwProg{Procedure}{procedure}{}{}
  \SetKwFunction{Configurations}{\textsc{Configurations}}
  \SetKwFunction{Head}{head}
  \SetKwFunction{Type}{type}
  \SetKwFunction{Last}{last}
  \SetKwFunction{Enum}{enum}
  \SetKwFunction{OrdTypes}{ordered-types}
  \Procedure{\Configurations{$C_{V}$}}{
    \KwIn{A complex event $C_{V} = \{i, \ldots, j\}$ with $C_{V} \subseteq 2^{\mathbb{N}}$.}
    \KwOut{A set $\mathcal{K}$ of configurations $K := \textbf{T} \times \mathbb{N}$ where $K$ is the mapping from the event type $t \in \textbf{T}$ to the size of the group of consecutive events of type $t$ in the complex $C_{V}$.}
    $\mathcal{V} \leftarrow \emptyset$\;
    $i \cup C_{V}' \leftarrow \Head{$C_{V}$}$\;
    $A \leftarrow \{ i \}$\;
    $\Type{$A$} \leftarrow \Type{$i$}$\;
    \For{$j \in C_{V}'$}{
      \eIf{$\Type{$j$} = \Type{$A$}$}{
        $A \leftarrow A \cup j$\;
        \uIf{$\Last{$C_{V}$} = j$} {
          $\mathcal{V} \leftarrow \mathcal{V} \cup \Enum{$1, |A|$}$
        }
      }{
        $\mathcal{V} \leftarrow \mathcal{V} \cup \Enum{$1, |A|$}$\;
        $A \leftarrow \{ j \}$\;
        $\Type{$A$} \leftarrow \Type{$j$}$\;
      }
    }
    $\mathcal{W} \leftarrow \bigtimes\limits_{V \in \mathcal{V}} V$\;
    $T \leftarrow \OrdTypes{$C_{V}$}$\;
    $\mathcal{K} \leftarrow \emptyset$\;
    \ForEach{$W \in \mathcal{W}$}{
      $K \leftarrow \emptyset$\;
      \For{$i \leftarrow 1$ \KwTo $|W|$}{
        $K \leftarrow K \cup (T[i], W[i])$\;
      }
      $\mathcal{C} \leftarrow \mathcal{C} \cup C$\;
    }
    \Return{$\mathcal{C}$}
  }
\caption{Configuration of a complex event $C_{V}$.}
\label{algo:configurations}
\end{algorithm}


\textsc{GroupBy} of line~\ref{algo:mcee:line:groupby} groups the set of tuples $\mathcal{K} := \{\langle C_{V}, K \rangle, \ldots\}$ by their configuration resulting in the set of tuples $D := \{\langle K, \mathcal{C}_{V}\} \rangle, \ldots \}$, where $\mathcal{C}_{V} := \{C^{1}_{V}, \ldots, C^{m}_{V}\}$. Then, the set $D$ is distributed among the $|\mathcal{P}|$ processing units using a generic load-balancing algorithm \textsc{Distribute}. The choice of implementation does not affect the correctness of the algorithm.

The procedure \textsc{Enumerate} receives the set of tuples $D := \{\langle K, \mathcal{C}_{V}\} \rangle, \ldots \}$ and enumerates all complex events included in each $\mathcal{C}_{V}$ filtered by $K$. $K$ is a configuration and for each event type $T$, it returns the exact number of complex event of that type that must be present in the resulting complex events. This is what allows us to control the load-balancing of the process, by distributing the configurations assigned to each process. For each tuple $\langle K, \mathcal{C}_{V} \rangle$, lines~10-13 are executed. First, a new \emph{tree} root $T$ is created on line~\ref{algo:mcee:line:new-node}. This $n$-ary tree will be used through the algorithm to detect which complex events have been outputted before to avoid duplicates. Then, for each maximal complex event $C_{V} \in \mathcal{C}_{V}$, the procedure \textsc{Partition} is executed and the result is given to procedure \textsc{Loop}.

The procedure \textsc{Partition} partitions the complex event $C_{V}$ in sets of consecutive positions of events of the same type. For example, \newline
\hspace*{60pt}$\textsc{Partition}(\{T, T, H, H, H\}) = \{\{T,T\}, \{H,H,H\}\}$

\textsc{Loop} receives as input a node $n$, a grouped complex event $\mathcal{G}$, a partial complex event $C_{V}$, a boolean $new$, and a configuration $K$. On each iteration, \textsc{Loop} extends $C_{V}$ with the next group in $\mathcal{G}$ and the configuration $K$ associated to that group. The \code{switch} from line~\ref{algo:mcee:line:switch} is split in two cases. Case 1 (lines~17-19) corresponds to the base case when $\mathcal{G}$ is empty. If the complex event $C_{V}$ has not been outputted before (i.e., $new = \top$), then we output the complex event $C_{V}$ and stop. Case 2 (lines~20-29) corresponds to the inductive step when $\mathcal{G}$ has at least a group of events of the same type in the corresponding complex event. Notice, that $\mathcal{G}$ on each iteration is smaller, ergo the algorithm terminates. For each group of events of the same type $G$, $k \in \mathbb{N}$ is retrieved from the configuration $K$, which corresponds to the size assigned to that processing unit for the group $G$. Different processing units will have different sizes assigned to each group, resulting in disjoint complex events enumerated by each unit. The $k$-combination set $N := \binom{G}{k}$ is computed, where $N$ contains permutations of complex event positions. Then, for each $k$-combinations of events, lines~23-29 are executed. In both cases, $C_{V}$ is extended with positions $i \subseteq 2^{\mathbb{N}}$. If there exists a children node $n'$ in $n$ that contains events $i$, then we recursively call \textsc{Loop} with extended complex even $C_{V} \cup i$, but we do not update $new$ since an event such as $C_{V} \cup i$ has already been outputted before. Otherwise, we create a new node $n'$ with events $i$, extend our current node $n$ with $n'$, and, as before, we call \textsc{Loop}, but this time with argument $new = \top$ indicating that complex event $C_{V}$ has not been enumerated before, which will eventually reach the base case and enumerate this new complex event.
