
\documentclass[12pt, a4paper]{article}

%
% Preamble.tex
%

\usepackage[utf8]{inputenc} % Input encoding

\usepackage[english]{babel} % use special characters and also translates some elements within the document.

\usepackage[numbers]{natbib} % Bibliography

\usepackage[nottoc]{tocbibind} % Adds bibliography, figures to ToC

% Hyperlinks \url{url} or \href{url}{name}
\usepackage[colorlinks = true,
            linkcolor = blue,
            urlcolor  = blue,
            citecolor = blue,
            anchorcolor = blue]{hyperref}

\usepackage{bm, stmaryrd} % math symbols

% Defines a new environment to write your or claim - proof
\newenvironment{claim}[1]{\par\noindent\underline{Claim:}\space#1}{}
\newenvironment{claimproof}[1]{\par\noindent\underline{Proof:}\space#1}{\hfill $\blacksquare$}

\usepackage{graphicx}
\graphicspath{{./figures/}}

% \begin{figure}[h]
%   \centering
%   \includegraphics[scale=0.5]{cat}  % [width=\textwidth, height=4cm],
%   \caption{Example of a cat}
%   \label{fig:cat}
% \end{figure}

\usepackage{tikz}
\usetikzlibrary{external, automata, arrows.meta, positioning, calc}
\tikzexternalize[shell escape=-shell-escape,prefix=tikz/cache/]

\newcommand{\inputtikz}[1]{%
  \tikzsetnextfilename{#1}%
  \input{tikz/#1.tex}
}

\usepackage{booktabs} % better default table

% \begin{table}
% \centering
% \caption{This is my table, there are many like it, but this one is mine.}
% \label{tbl:mytable}
% \begin{tabular}{llr}
% \toprule
% \multicolumn{2}{c}{Item} \\
% \cmidrule(r){1-2}
% Animal & Description & Price (\$) \\
% \midrule
% Gnat  & per gram & 13.65 \\
%       & each     &  0.01 \\
% Gnu   & stuffed  & 92.50 \\
% Emu   & stuffed  & 33.33 \\
% Armadillo & frozen & 8.99 \\
% \bottomrule
% \end{tabular}
% \end{table}

\usepackage{minted}
\usemintedstyle{default}
\newminted{scala}{frame=lines, framerule=2pt}
% \mint{html}|<h2>Something <b>here</b></h2>|
% \inputminted{octave}{BitXorMatrix.m}

%\begin{listing}[H]
  %\begin{minted}[xleftmargin=20pt,linenos,bgcolor=codegray]{haskell}
  %\end{minted}
  %\caption{Example of a listing.}
  %\label{lst:example} % You can reference it by \ref{lst:example}
%\end{listing}


\usepackage[vlined,ruled]{algorithm2e} % pseudo-code
% \newcommand\commentfont[1]{\footnotesize\ttfamily\colorbox{gray}{\textcolor{white}{#1}}}
\newcommand\commentfont[1]{\small\ttfamily\textcolor{darkgray}{#1}}
\SetCommentSty{commentfont}


\usepackage{xcolor} % \definecolor, \color{codegray}
\definecolor{codegray}{rgb}{0.9, 0.9, 0.9}
% \color{codegray} ... ...
% \textcolor{red}{easily}

\usepackage{enumitem} % \begin{enumerate}[label=(\alph*)]

\usepackage[titletoc]{appendix}

%%%% Glossaries & Acronyms

\usepackage{glossaries}

\glsdisablehyper % remove hyperlinks to glossaries
\setacronymstyle{long-short} %short-long

\newacronym{tecs}{tECS}{\emph{timed Enumerable Compact Set}}
\newacronym{core}{CORE}{Complex Event Recognition Engine}
\newacronym{mmde}{\textsc{MMDE}}{\emph{Maximal Match Disjoint Enumeration}}
\newacronym{dte}{\textsc{DTE}}{\emph{Distributed tECS Enumeration}}
\newacronym{cea}{CEA}{Complex Event Automata}
\newacronym{rmi}{RMI}{Remote Method Invocation}
\newacronym{dag}{DAG}{Directed Acyclic Graph}
\newacronym{cer}{CER}{Complex Event Recognition}
\newacronym{cep}{CEP}{Complex Event Processing}
\newacronym{cel}{CEL}{\emph{Complex Event Logic}}
\newacronym{socel}{SO-CEL}{Set-Oriented Complex Event Logic}

%%%%%% Subfigure

\usepackage{subcaption} % \subfigure
\usepackage{pgffor} % \foreach
\usepackage{fp}

\def\makeTECS#1#2{%
  \foreach \index in {0, ..., #2} {%
    \FPset\x{#2}
    \FPeval\y{1.0/(x + 1.0) - 0.05}
    \begin{subfigure}[b]{\y\textwidth}
      \centering
      \inputtikz{#1_\index}
      \caption*{S[\index]}
    \end{subfigure}
  }}

%%%% New commands

\newcommand{\comment}[1]{} % \comment{this will not appear in the document}
\newcommand{\cea}{CEA $\mathcal{A} = (Q, \Delta, q_{0}, F)$}
\newcommand{\enumCEA}{${\llbracket \mathcal{A} \rrbracket}^{\epsilon}_{j}(S)$}
\newcommand{\enumNode}{${\llbracket \text{n} \rrbracket}^{\epsilon}_{\mathcal{E}}(j)$}
\newcommand{\tecs}{${\mathcal{E}}$}
\newcommand{\InDoubleBrackets}[1]{\llbracket #1 \rrbracket}

\newcommand{\code}[1]{\texttt{#1}} % \code{foo.hs} environment

\usepackage{fancyhdr}
\pagestyle{fancy}
\fancyhf{}
\rhead{TODO}
\lhead{TODO}
\rfoot{Page \thepage}

\title{%
  \vspace{-10ex}
  Draft: Distributed Enumeration
}
\author{%
  Arnau Abella \\
  \large{Universitat Polit\`ecnica de Catalunya}
}
\date{\today}

\begin{document}
\maketitle
% \tableofcontents

%%%%%%%%%%%%%% DOCUMENT %%%%%%%%%%%%%%%%%%%%%%

\section{Introduction}
\label{sec:introduction}

On one hand, CORE's evaluation algorithm guarantees \textit{output-linear} delay enumeration of the output \cite{core}.
Although each enumeration can be accomplished in linear time with respect to the size of the complex event,
the cost of enumerating all complex events from a CEQL that contains an iteration is exponential \textit{w.r.t} the size of this clause \cite{grez_et_al:LIPIcs:2019:10307}.

Selection strategies (\textit{STRICT}, \textit{NXT}, \textit{LAST}, \textit{MAX}) were created to restrict the size of the output set.
Selection strategies have low impact in CORE's update and enumeration time complexity since the partial output set is already represented in a compact data but, they impact the number of complex events found in a stream and therefore the number of enumerations.

We observed, that in the presence of iterations, all previous for strategies still suffer exponential growth in the number of complex events to be enumerated.

\section{Algorithms for distributed enumeration}
\label{sec:algorithm}

In order to deal with the exponential complexity of enumerating $\varphi_{1} ; \varphi_{2}$ (in particular $\varphi+$), we propose the following distributed enumeration algorithms Algorithm \ref{algo:mmde} and Algorithm \ref{algo:de}.

\subsection{Maximal Matches Disjoint Enumeration}
\label{subsec:mmde}

The algorithms presented in \textit{CORE: a COmplex event Recognition Engine} are based on the well-founded complex event semantics $\llbracket \varphi \rrbracket (S) := \{ C_{V} \ | \ V \in \llceil \varphi \rrfloor (S) \}$. The algorithms of this section are based on the definition of match (\ref{def:match}) and maximal match (\ref{def:maximalmatch}) which are both data structures produced by CORE.

\begin{definition}[Match]
  \label{def:match}
  TODO
\end{definition}

\begin{definition}[Maximal Match]
  \label{def:maximalmatch}
  TODO
\end{definition}

\begin{algorithm}[H]
  \DontPrintSemicolon
  \SetAlgoNoEnd % don't print end
  \SetAlgoNoLine % no vertical lines
  \LinesNumbered
  \SetKwProg{Procedure}{procedure}{}{}
  \SetKwFunction{MMDE}{\textsc{MaximalMatchesDisjointEnumeration}}
  \SetKwFunction{Enumerate}{\textsc{Enumerate}}

  \Procedure{\MMDE{$M$, $W$}}{
    \KwIn{A set of maximal matches $M := \{M_{1}, \ldots, M_{n}\}$ \newline
      and a set of workers $W := \{w_{1},\ldots, w_{m}\}$.
    }
    \KwResult{Enumerates all \textit{submatches} $\subseteq M$ without repetitions.}
    $C \leftarrow \emptyset$\;
    \ForEach{$M_{i} \in M$}{
        $C \leftarrow C \cup \textsc{Configurations}(M_{i}).map(\lambda c \to ( c, M_{i} ))$\;
    }
    $D \leftarrow C.groupBy(\lambda (c, \_ ) \to c)$\;
    $\textsc{Distribute}(W, D)$\tcp*[l]{\textit{Round-robin, Bins \& Balls, etc.}}
    \ForEach{$w \in W$}{
      \tcp{\textsc{Enumerate} is executed in a remote machine}
      $w.\textsc{Enumerate}()$
    }
  }
  \;
\caption{Non-repeated enumeration of a set of maximal matches.}
\label{algo:mmde}
\end{algorithm}

% This procedure enumerates all submatches of M without repetitions.
% It stills enumerates all submatches but only outputs non-repeated.
% It efficiently detects repetitions by constructing an n-ary tree of complex events.
% The complexity is still exponential w.r.t. the size of the largest iteration.
% The exponential time enumeration must be repeated a constant factor of times.
\begin{algorithm}[H]
  \DontPrintSemicolon
  \SetAlgoNoEnd % don't print end
  \SetAlgoNoLine % no vertical lines
  \LinesNumbered
  \SetKwProg{Procedure}{procedure}{}{}
  \SetKwFunction{Enumerate}{\textsc{Enumerate}}
  \SetKwFunction{Enumeratee}{\textsc{Enumerate'}}

  \Procedure{\Enumerate{}}{
    \KwData{A set of tuples $A = \{ (c, \{ M_{1}, \ldots, M_{n}\}) \}$ where $c$ is a \textit{configuration} and $M_{i}$ are maximal matches.}
    \KwOut{The set of all submatches without repetitions.}
    \ForEach{$(c, M) \in A$}{
      $T \leftarrow$ \textit{newRoot()}\tcp*[l]{Root of a \textit{n}-ary tree}
      \ForEach{$M_{i} \in M$}{
        $G \leftarrow \textsc{GroupBy}(M_{i})$\tcp*[l]{$\textsc{GroupBy}(A_{1}A_{2}B_{1}B_{2}C_{1}) = \{ \{A_{1}, A_{2}\}, \{B_{1}, B_{2}\}, \{C_{1}\} \}$}
        }
        $\textsc{Enumerate'}(T, G, \emptyset, \bot)$
    }
  }
  \;
  \Procedure{\Enumeratee{$n, G, S, new$}}{
    \KwData{A node $n$, a set of grouped events $G$, a time-ordered set of events $S$, and a boolean $new$.}
    \Switch{$G$}{
      \uCase{$\emptyset$}{
        \If{$new$}{
          \Return{$S$}
        }
      }
      \uCase(\tcp*[h]{$e.g. \ g = \{A_{1}, A_{2}\}, \ g.type = A$}){$g \cup G'$}{
        $k \leftarrow c(g.type)$\tcp*[l]{$k \in \mathbb{N}$}
        $E \leftarrow \binom{g}{k}$\tcp*[l]{The \textit{k}-combination set}
        \ForEach{$e \in E$}{
          \eIf{$\exists n' \in n.children \land n'.event = e$}{
            $\textsc{Enumerate'}(n', G', S \cup e, new)$\;
          }{
            $p \leftarrow \text{\textit{new-node}}(e)$\;
            $n.children.add(p)$\;
            $\textsc{Enumerate'}(p, G', S \cup e, \top)$\;
          }
        }
      }
    }
  }
  \;
\caption{Non-repeated enumeration of a set of maximal matches given a predicate configuration.}
\label{algo:enumerate}
\end{algorithm}

\begin{algorithm}[H]
  \DontPrintSemicolon
  \SetAlgoNoEnd % don't print end
  \SetAlgoNoLine % no vertical lines
  \LinesNumbered
  \SetKwProg{Procedure}{procedure}{}{}
  \SetKwFunction{Configurations}{\textsc{Configurations}}

  \Procedure{\Configurations{$M$}}{
    \KwIn{A match $M = \{e_{1}, \ldots, e_{n}\}$ where $e_{i}$ is an event of type $t \in T$.}
    \KwOut{A set $C$ of configurations $c := T \times \mathbb{N}$ where $c$ is the mapping from the event type $t \in T$ to the size of the iteration of the event type $t$ in the submatches of $M$.}
    $V \leftarrow newList$\;
    $e_{0} \cup M' \leftarrow pop(M)$\;
    $A \leftarrow \{ e_{0} \}$\;
    $A.type \leftarrow e_{0}.type$\;
    \For{event $e$ in $M'$}{
      \eIf{$e.type = A.type$}{
        $A \leftarrow A \cup e$\;
        \uIf{$isLast(e)$} {
          $V \leftarrow V + enumFromTo(1, |A|)$
        }
      }{
        $V \leftarrow V + enumFromTo(1, |A|)$\;
        $A \leftarrow \{ e \}$\;
        $A.type \leftarrow e.type$\;
      }
    }
    $WW \leftarrow V_{1} \times \cdots \times V_{n}$\tcp*[l]{$V = \{V_{1}, \cdots, V_{n}\}$}
    $T \leftarrow types(M)$\tcp*[l]{Ordered set of types e.g. $types(A_{1}A_{2}B_{1}C_{1}) = \{A,B,C\}$}
    $C \leftarrow \emptyset$\;
    \ForEach(\tcp*[h]{E.g. $W = \{1, 2, 1\}$}){$W \in WW$}{
      $c \leftarrow \emptyset$\tcp*[l]{E.g. $c = \{(A,1), (B,2), (C, 1)\}$}
      \For{$i \leftarrow 1$ \KwTo $|W|$}{
        $c \leftarrow c \cup (T[i], W[i])$\;
      }
      $C \leftarrow C \cup c$\;
    }
    \Return{C}
  }
\caption{Computes all disjoint configurations of a maximal match.}
\label{algo:configurations}
\end{algorithm}

\subsection{Proof}
\label{subsec:proof}

% You need to make the following observations of "Maximal Matches Enumeration":
% 1. The algorithm produces disjoint submatches given a maximal match.
% 2. The algorithm produces non-disjoint submatches given multiple maximal matches.

% But (2) can be analyzed further:
% 1. Disjoint configurations produce disjoint submatches.
% 2. Non-disjoint configurations produce non-disjoint submatches.

% From previous observations we can conclude that repeated submatches are only generated by applying the same configuration to different maximal matches.

% Uniqueness of submatches is guaranteed by (3) and (4).
% (3) guarantees that the output of each worker is disjoint wrt the others.
% (4) guarantees that the output of a worker is disjoint.

% The complexity of the algorithm remains the same if we accomplish linear time enumeration in each worker (this is the tricky part).

\begin{lemma}[TODO]
  \label{lemma:todo}
  TODO
\end{lemma}

\begin{theorem}[TODO]
  \label{theorem:todo}
  TODO
\end{theorem}

\begin{example}[TODO]
  \label{example:todo}
  TODO
\end{example}

\subsection{Distributed tECS enumeration}
\label{subsec:label}

\begin{algorithm}[H]
  \DontPrintSemicolon
  \SetAlgoNoEnd % don't print end
  \SetAlgoNoLine % no vertical lines
  \LinesNumbered
  \SetKwProg{Procedure}{procedure}{}{}
  \SetKwFunction{Distribute}{\textsc{Distribute}}
  \SetKwFunction{Enumerate}{\textsc{Enumerate}}

  \Procedure{\Distribute{$\mathcal{E}, n, \epsilon, j, P, W$}}{
     \nl $b <- bitstring(|P|)$\;
     \nl $B <- permutations(b)$\;
     $\mathcal{G} \leftarrow \emptyset$\;
     \ForEach{$b \in B$}{
       $G = \emptyset$\tcp*[l]{$f: N_{U} \rightarrow \{ \bot , \top \}$}
       \For{$i \leftarrow 1$ \KwTo $|p|$}{
           $G \leftarrow G \cup (P_{i}, b_{i})$\;\label{algo:de:7}
       }
       $\mathcal{G} \leftarrow \mathcal{G} \cup G$\;
     }
     $S \leftarrow \emptyset$\tcp*[l]{Assignments}\label{algo:de:9}
     \For{$i \leftarrow 1$ \KwTo $|\mathcal{G}|$}{
       \nllabel{2}$i' \leftarrow i \mod |W|$\;
       $S \leftarrow S \cup (S_{i'} \cup \{ \mathcal{G}_{i} \} )$\;
     }\label{algo:de:12}
     \For{$i \leftarrow 1$ \KwTo $|W|$}{
        $W_{i}.\textsc{Enumerate}(\mathcal{E}, n, \epsilon, j, S_{i})$\;
     }
  }
\caption{Distributed enumeration of $\llbracket n \rrbracket^{\epsilon}_{\mathcal{E}}(j)$}
\label{algo:de}
\end{algorithm}

\setlength{\interspacetitleruled}{-.4pt}%
\begin{algorithm}[H]
  \DontPrintSemicolon
  \SetAlgoNoEnd % don't print end
  \SetAlgoNoLine % no vertical lines
  \LinesNotNumbered
  \SetKwProg{Procedure}{procedure}{}{}
  \SetKwFunction{Enumerate}{\textsc{Enumerate}}
  \nl \Procedure{\Enumerate{$\mathcal{E}, n, \epsilon, j, \mathcal{G}$}}{
    $\cdots$\;
    \setcounter{AlgoLine}{13}
    \nl \ElseIf{$n' \in N_{U}$}{
      \nl$choice \leftarrow (\bot, \bot)$\tcp*[l]{(left, right)}
      \nl\ForEach{$G \in \mathcal{G}$}{\label{algo:enumerate:16}
        \nl\If{$G_{1}(u)$}{
          \nl$choice.right = \top$\;
        }
        \nl\Else{
          \nl$choice.left = \top$\;
        }
        \nl$G \leftarrow G \setminus \{ G_{i} \}$\;\label{algo:enumerate:21}
      }
      \nl\If{$choice.right \land max(right(n')) \geq \tau$}{\label{algo:enumerate:22}
        \nl$push(st, (right(n'), P))$\;
      }
      \nl\ElseIf{$choice.left$}{\label{algo:enumerate:24}
        \nl$n' \leftarrow left(n')$\;
      }
    }
    $\cdots$\;
  }
\label{algo:enumerate}
\end{algorithm}

We present Algorithm \ref{algo:de} which distributes the workload of \textsc{Enumerate} (Algorithm 1) from \cite{core} among $N$ workers. In order to incrementally distribute the enumeration of $\mathcal{E}$, we need to manipulate an additional ordered set $P$ of union nodes during the evaluation of \textsc{Evaluation}. This set will store all unions nodes present in $\mathcal{E}$ during the evaluation of the \gls{cea}. The algorithm consist of two distinct procedures: \textsc{Distribute} and \textsc{Enumerate}. \textsc{Distribute} replaces \textsc{Enumerate} in the original algorithm and \textsc{Enumerate} is executed remotely on each worker.

In detail, Algorithm \ref{algo:de} receives as an input a gls{tecs} $\mathcal{E}$, a node $n$, a time-window bound $\epsilon$, a position $j$, an ordered set of union nodes $P$ and, a set of workers $W$. First, creates a bitstring (e.g. $111111$) of size $|p|$ where each position corresponds to a union node in $\mathcal{E}$ and computes the $2^{n}$ permutations of this bitstring i.e. all possible paths that traverses the \gls{dag} $G_{\mathcal{E}}$ encoded by $\mathcal{E}$. Then, for each possible path, line \ref{algo:de:7} computes the mapping between the union node $P_{i}$ and the choice $b_{i}$ ($\top \equiv \text{ right}$ and $\bot \equiv \text{ left}$). Then, lines \ref{algo:de:9}-\ref{algo:de:12} distribute the paths among all workers. We chose the simples load-balancing algorithm, the round-robin but these lines can be replaced by any load-balancing algorithm. Finally, we call \textsc{Enumerate} on each worker using a \gls{rmi}.

On the other side of the cluster, each worker will be waiting for the remote call \textsc{Enumerate}. \textsc{Enumerate} is a modification of the algorithm \textsc{Enumerate} (Algorithm 2) from \cite{core}. The procedure receives a set $\mathcal{G}$ of binary relations $G$ that are the choices to be made on each union node. Notice, the traversing of the \gls{dag} can be done in a single traversal if we combine all the path choices at each union node (lines \ref{algo:enumerate:16}-\ref{algo:enumerate:21}). Then lines \ref{algo:enumerate:22} and \ref{algo:enumerate:24} have been modified to accommodate this choice.

(we need to prove this, it should be easy) Algorithm \ref{algo:de} still enumerates the set ${\llbracket n \rrbracket}^{\epsilon}_{\mathcal{E}}(j)$ in output-liner delay and without repetitions.

%%%%%%%%%%%%%% BIBLIOGRAPHY %%%%%%%%%%%%%%%%%%%%%%

\bibliographystyle{abbrv}
\bibliography{report}
%
% \cite{einstein}
% \citet{key}               Jones et al. (1990)
% \citet*{key}              Jones, Baker, and Smith (1990)
% \citep{key}               (Jones et al. 1990)
% \citep*{key}              (Jones, Baker, and Smith 1990)
% \citep[p.~99]{key}        (Jones et al., 1990, p. 99)
% \citep[e.g.][]{key}       (e.g. Jones et al., 1990)
% \citep[e.g.][p.~99]{key}  (e.g. Jones et al., 1990, p. 99)
% \citeauthor{key}          Jones et al.
% \citeauthor*{key}         Jones, Baker, and Smith
% \citeyear{key}            1990
\end{document}
