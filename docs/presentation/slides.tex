\documentclass[xcolor=pdftex,dvipsnames,table]{beamer}

\mode<presentation>
{
  \usetheme{metropolis}
  \usecolortheme{default}
}

%%%%%%%%%%%% PACKAGES %%%%%%%%%%%%%%%%%%

\usepackage[utf8]{inputenc}

\usepackage[english]{babel}

\usepackage{amsmath,amsfonts,amssymb,amscd,amsthm, bm, stmaryrd}

\usepackage{listings}
\usepackage[cache=false,section]{minted}

\usepackage{enumitem}

\usepackage{multirow}

\usepackage{tikz}
\usetikzlibrary{external, automata, arrows.meta, positioning, calc}
\tikzexternalize[shell escape=-shell-escape,mode=graphics if exists, prefix=tikz/cache/]
\newcommand{\inputtikz}[1]{%
  \tikzsetnextfilename{#1}%
  \input{tikz/#1.tex}
}

\usepackage{graphicx}
\graphicspath{{./figures/}}

\usepackage{xcolor,colortbl}
\definecolor{Red}{rgb}{0.89, 0.0, 0.13}
\definecolor{Pink}{rgb}{1.0, 0.65, 0.79}
\definecolor{lightgreen}{rgb}{0.56, 0.93, 0.56}
\newcolumntype{a}{>{\columncolor{lightgreen}}c}
\newcolumntype{b}{>{\columncolor{Red}}c}


%%%%%%%%%%%%% TITLE %%%%%%%%%%%%%%%%%

\title{Distributed Complex Event Recognition}
\subtitle{Master's Thesis\\
  \tiny Master in Innovation and Research in Informatics
}
\author{Arnau Abella}
\institute[UPC]{%
  {\tiny %
   \textit{Supervisors:}
   \begin{itemize}[leftmargin=5pt]
     \item[] Sergi Nadal, Universitat Politècnica de Catalunya
     \item[] Stijn Vansummeren, UHasselt – Hasselt University
   \end{itemize}
  }
  \vspace{10pt}
  \textrm{\scriptsize%
    Facultat d’Informàtica de Barcelona (FIB)\\
    Universitat Politècnica de Catalunya (UPC)\\
  }
}
\date{\tiny \today}

%%%%%%%%%%%% DOCUMENT %%%%%%%%%%%%%%%%%%

\AtBeginSection[]
{
  \begin{frame}<beamer>
      \frametitle{Table of Contents}
      \tableofcontents[currentsection]
  \end{frame}
}

\begin{document}

\frame{\titlepage}

\section{Introduction}
% Explain CER with example 1.1.

\begin{frame}[allowframebreaks, fragile]{Complex Event Recognition}
  \begin{example}
    A stream produced by wireless sensors placed in a warehouse, whose main objective is to detect fires.
    \begin{figure}[H]
      \centering
      \begin{tabular}{|c|c|c|c|c|c|c|c|c|c|c}\hline
        type  &$H$&$T$&$H$&$H$&$T$&$H$&$H$&$T$&$T$ & \ldots \\ \hline
        id  & 1 & 1 & 2 & 1 & 2 & 2 & 1 & 1 & 1 & \multirow{2}{*}{\ldots} \\
        val & 50 & 24& 49& 24& 24& 42& 23& 40& 45\\ \hline
        timestamp & 0 & 1 & 2 & 3 & 4 & 5 & 6 & 7 & 8 & \ldots \\ \hline
      \end{tabular}
      \caption{Exemplary stream of events measuring temperature ($T$) and relative humidity ($H$)}
      \label{fig:stream}
    \end{figure}
  \end{example}
  \begin{block}{}

    When the temperature of a storage room increases from below 30 celsius degrees to above 40 celsius degrees and the humidity is below 25\% there is a high probability of fire.

    \begin{figure}[H]
      \begin{minted}[xleftmargin=60pt, linenos=false, fontsize=\footnotesize]{text}
SELECT t2.id FROM warehouse
WHERE (T as t1; H as h1; T as t2)
FILTER t1[val < 30] AND h1[val < 25]
  AND t2[val > 40] AND t1[id] = h1[id]
  AND h1[id] = t2[id]
WITHIN 10 events
      \end{minted}
      \caption{Query on a wireless sensors network stream, which goal is to detect fires.}
      \label{fig:query:1}
    \end{figure}
  \end{block}

 \begin{block}{}
    \begin{figure}[H]
      \centering
      \begin{tabular}{|c|c|a|c|a|c|c|c|a|c|c}\hline
        type  &$H$&$T$&$H$&$H$&$T$&$H$&$H$&$T$&$T$ & \ldots \\ \hline
        id  & 1 & 1 & 2 & 1 & 2 & 2 & 1 & 1 & 1 & \multirow{2}{*}{\ldots} \\
        val & 50 & 24& 49& 24& 24& 42& 23& 40& 45\\ \hline
        timestamp & 0 & 1 & 2 & 3 & 4 & 5 & 6 & 7 & 8 & \ldots \\ \hline
      \end{tabular}
      \caption{Exemplary stream of events measuring temperature ($T$) and relative humidity ($H$)}
      \label{fig:stream}
    \end{figure}
 \end{block}

 \begin{block}{}
    \begin{figure}[H]
      \centering
      \begin{tabular}{|c|c|a|c|c|c|c|a|a|c|c}\hline
        type  &$H$&$T$&$H$&$H$&$T$&$H$&$H$&$T$&$T$ & \ldots \\ \hline
        id  & 1 & 1 & 2 & 1 & 2 & 2 & 1 & 1 & 1 & \multirow{2}{*}{\ldots} \\
        val & 50 & 24& 49& 24& 24& 42& 23& 40& 45\\ \hline
        timestamp & 0 & 1 & 2 & 3 & 4 & 5 & 6 & 7 & 8 & \ldots \\ \hline
      \end{tabular}
      \caption{Exemplary stream of events measuring temperature ($T$) and relative humidity ($H$)}
      \label{fig:stream}
    \end{figure}
 \end{block}

 \begin{block}{}
    \begin{figure}[H]
      \centering
      \begin{tabular}{|c|c|a|c|a|c|c|c|c|a|c}\hline
        type  &$H$&$T$&$H$&$H$&$T$&$H$&$H$&$T$&$T$ & \ldots \\ \hline
        id  & 1 & 1 & 2 & 1 & 2 & 2 & 1 & 1 & 1 & \multirow{2}{*}{\ldots} \\
        val & 50 & 24& 49& 24& 24& 42& 23& 40& 45\\ \hline
        timestamp & 0 & 1 & 2 & 3 & 4 & 5 & 6 & 7 & 8 & \ldots \\ \hline
      \end{tabular}
      \caption{Exemplary stream of events measuring temperature ($T$) and relative humidity ($H$)}
      \label{fig:stream}
    \end{figure}
 \end{block}

 \begin{block}{}
    \begin{figure}[H]
      \centering
      \begin{tabular}{|c|c|a|c|c|c|c|a|c|a|c}\hline
        type  &$H$&$T$&$H$&$H$&$T$&$H$&$H$&$T$&$T$ & \ldots \\ \hline
        id  & 1 & 1 & 2 & 1 & 2 & 2 & 1 & 1 & 1 & \multirow{2}{*}{\ldots} \\
        val & 50 & 24& 49& 24& 24& 42& 23& 40& 45\\ \hline
        timestamp & 0 & 1 & 2 & 3 & 4 & 5 & 6 & 7 & 8 & \ldots \\ \hline
      \end{tabular}
      \caption{Exemplary stream of events measuring temperature ($T$) and relative humidity ($H$)}
      \label{fig:stream}
    \end{figure}
 \end{block}
\end{frame}








% Explain issue with CER
% Introduce CORE
% Explain issues with CORE
% Contributions

\begin{frame}{Foo}
  \begin{block}{}

  \end{block}
\end{frame}

\section{Preliminaries}
% Introduce Events, complex events, valuations, etc.
% Introduce CEQL syntax
% Introduce CEL
% Explain evaluation of CEQL
% Explain selection strategies
% Introduce CEA

\section{Distributed CER}
% Introduce distributed CER framework (figure would be enough)
% Introduce DCERE
% Introduce distribution strategies
% Introduce DCORE

\section{Distributed Evaluation Algorithm}
% Explain goal
% Explain tECS
% Explain union lists
% Explain hash tables
% Explain evaluation algorithm by means of figures
% Explain enumeration procedure by means of figures
% Do not mention theorems nor pseudocode

\section{Experiments}
% Explain CORE and DCORE implementation
% Explain settings of the experiments
% Explain experiments and results
% Maybe CV can be removed

\section{Conclusions and Future Work}
% Just copy/paste Contributions I and II
% Explain future work with a list

\begin{frame}{Remove me}
  \begin{block}{Remove me}
    \cite{core}
  \end{block}
\end{frame}

\begin{frame}[allowframebreaks]
  \frametitle{Bibliography}
  \bibliographystyle{unsrtnat}
  \bibliography{bibliography}
\end{frame}

\end{document}

