\documentclass[xcolor=pdftex,dvipsnames,table]{beamer}

\mode<presentation>
{
  \usetheme{Madrid}
  \usecolortheme{seagull}
  \setbeamercovered{dynamic} %invisible,transparent,dynamic,highly dynamic
}

%%%%%%%%%%%%%%%%%%%%%%%%%%%%%%%%%%%%%%%%
%%%%%%%%%%%% PACKAGES %%%%%%%%%%%%%%%%%%

\usepackage[utf8]{inputenc}

\usepackage[english]{babel}

\usepackage{amsmath,amsfonts,amssymb,amscd,amsthm, bm, stmaryrd}

\usepackage{listings}
\usepackage[cache=false,section]{minted}


\usepackage{multirow}

\usepackage{caption}
\usepackage{subcaption}

\usepackage{tikz}
\usetikzlibrary{external, automata, arrows.meta, positioning, calc}
\tikzexternalize[shell escape=-shell-escape,mode=graphics if exists, prefix=../thesis/tikz/cache/]
\newcommand{\inputtikz}[1]{%
  \tikzsetnextfilename{#1}%
  \input{../thesis/tikz/#1.tex}
}

\usepackage{graphicx}
\graphicspath{{../thesis/figures/}}

\usepackage{xcolor,colortbl}
\definecolor{Red}{rgb}{0.89, 0.0, 0.13}
\definecolor{Pink}{rgb}{1.0, 0.65, 0.79}
\definecolor{lightgreen}{rgb}{0.56, 0.93, 0.56}
\newcolumntype{a}{>{\columncolor{lightgreen}}c}
\newcolumntype{b}{>{\columncolor{Red}}c}

\newcommand{\code}[1]{\texttt{#1}}

%%%%%%%%%%%%%%%%%%%%%%%%%%%%%%%%%%%%%%
%%%%%%%%%%%%% TITLE %%%%%%%%%%%%%%%%%

\title{Distributed Complex Event Recognition}
\subtitle{Master's Thesis}
\author{Arnau Abella}
\institute[UPC]{%
  {\tiny %
    \textit{Supervisors:}\\
    Sergi Nadal, Universitat Politècnica de Catalunya\\
    Stijn Vansummeren, UHasselt – Hasselt University\\
  }
  \vspace{10pt}
  \textrm{\scriptsize%
    Master in innovation and research in informatics\\
    \vspace{5pt}
    Facultat d’Informàtica de Barcelona (FIB)\\
    Universitat Politècnica de Catalunya (UPC)\\
  }
}
\date{\tiny \today}

%%%%%%%%%%%%%%%%%%%%%%%%%%%%%%%%%%%%%%%%
%%%%%%%%%%%% DOCUMENT %%%%%%%%%%%%%%%%%%

\AtBeginSection[]
{
  \begin{frame}<beamer>
      \frametitle{Table of Contents}
      \tableofcontents[currentsection]
  \end{frame}
}

\begin{document}

\frame{\titlepage}

%%%%%%%%%%%%%%%%%%%%%%%%%%%%%%%%%%%%%%%%%%%%%%%%%%%%%%%%%%%%%%%%%%
%%%%%%%%%%%%%%%%%%%%%%%%%%%%%%%%%%%%%%%%%%%%%%%%%%%%%%%%%%%%%%%%%%

\section{Introduction}

\begin{frame}[fragile]{Introduction}
  \begin{block}{\emph{Complex Event Recognition (CER)}}
    Identifying collections of events that collectively satisfy some pattern in high-velocity streams.
  \end{block}

  \begin{block}{Features}
    Allow expressing patterns based on:
   \begin{itemize}
     \item Content
     \item Position in the stream
     \item Order w.r.t other events
     % \item More constraints
   \end{itemize}
  \end{block}
\end{frame}

\begin{frame}[fragile]{Introduction}
  \begin{example}
    A stream produced by wireless sensors placed in a warehouse, whose main objective is to detect fires.
    \begin{figure}[H]
      \centering
      \begin{tabular}{|c|c|c|c|c|c|c|c|c|c|c}\hline
        type  &$H$&$T$&$H$&$H$&$T$&$H$&$H$&$T$&$T$ & \ldots \\ \hline
        id  & 1 & 1 & 2 & 1 & 2 & 2 & 1 & 1 & 1 & \multirow{2}{*}{\ldots} \\
        val & 50 & 24& 49& 24& 24& 42& 23& 40& 45\\ \hline
        timestamp & 0 & 1 & 2 & 3 & 4 & 5 & 6 & 7 & 8 & \ldots \\ \hline
      \end{tabular}
    \end{figure}
  \end{example}
\end{frame}

\begin{frame}[fragile]{Introduction}
  \begin{example}{}
    When the temperature of a storage room increases from below 30 celsius degrees to above 40 celsius degrees and the humidity is below 25\% there is a high probability of fire.
    \begin{figure}[H]
      \begin{minted}[xleftmargin=60pt, linenos=false, fontsize=\footnotesize]{text}
SELECT t2.id FROM warehouse
WHERE (T as t1; H as h1; T as t2)
FILTER t1[val < 30] AND h1[val < 25]
  AND t2[val > 40] AND t1[id] = h1[id]
  AND h1[id] = t2[id]
WITHIN 5 minutes
      \end{minted}
    \end{figure}
  \end{example}
\end{frame}

\begin{frame}[fragile]{Introduction}
 \begin{example}{}
    \begin{figure}[H]
      \begin{minted}[xleftmargin=80pt, linenos=false, fontsize=\footnotesize]{text}
SELECT t2.id FROM warehouse
WHERE (T as t1; H as h1; T as t2)
FILTER t1[val < 30] AND h1[val < 25]
  AND t2[val > 40] AND t1[id] = h1[id]
  AND h1[id] = t2[id]
WITHIN 10 events
      \end{minted}
    \end{figure}
    \begin{figure}[H]
      \centering
      \begin{tabular}{|c|c|c|c|c|c|c|c|c|c|c}\hline
        type  &$H$&$T$&$H$&$H$&$T$&$H$&$H$&$T$&$T$ & \ldots \\ \hline
        id  & 1 & 1 & 2 & 1 & 2 & 2 & 1 & 1 & 1 & \multirow{2}{*}{\ldots} \\
        val & 50 & 24& 49& 24& 24& 42& 23& 40& 45\\ \hline
        timestamp & 0 & 1 & 2 & 3 & 4 & 5 & 6 & 7 & 8 & \ldots \\ \hline
      \end{tabular}
    \end{figure}
 \end{example}
\end{frame}

\begin{frame}[fragile]{Introduction}
 \begin{example}{}
    \begin{figure}[H]
      \begin{minted}[xleftmargin=80pt, linenos=false, fontsize=\footnotesize]{text}
SELECT t2.id FROM warehouse
WHERE (T as t1; H as h1; T as t2)
FILTER t1[val < 30] AND h1[val < 25]
  AND t2[val > 40] AND t1[id] = h1[id]
  AND h1[id] = t2[id]
WITHIN 10 events
      \end{minted}
    \end{figure}
    \begin{figure}[H]
      \centering
      \begin{tabular}{|c|c|a|c|a|c|c|c|a|c|c}\hline
        type  &$H$&$T$&$H$&$H$&$T$&$H$&$H$&$T$&$T$ & \ldots \\ \hline
        id  & 1 & 1 & 2 & 1 & 2 & 2 & 1 & 1 & 1 & \multirow{2}{*}{\ldots} \\
        val & 50 & 24& 49& 24& 24& 42& 23& 40& 45\\ \hline
        timestamp & 0 & 1 & 2 & 3 & 4 & 5 & 6 & 7 & 8 & \ldots \\ \hline
      \end{tabular}
    \end{figure}
 \end{example}
\end{frame}

\begin{frame}[fragile]{Introduction}
 \begin{example}{}
    \begin{figure}[H]
      \begin{minted}[xleftmargin=80pt, linenos=false, fontsize=\footnotesize]{text}
SELECT t2.id FROM warehouse
WHERE (T as t1; H as h1; T as t2)
FILTER t1[val < 30] AND h1[val < 25]
  AND t2[val > 40] AND t1[id] = h1[id]
  AND h1[id] = t2[id]
WITHIN 10 events
      \end{minted}
    \end{figure}
    \begin{figure}[H]
      \centering
      \begin{tabular}{|c|c|a|c|c|c|c|a|a|c|c}\hline
        type  &$H$&$T$&$H$&$H$&$T$&$H$&$H$&$T$&$T$ & \ldots \\ \hline
        id  & 1 & 1 & 2 & 1 & 2 & 2 & 1 & 1 & 1 & \multirow{2}{*}{\ldots} \\
        val & 50 & 24& 49& 24& 24& 42& 23& 40& 45\\ \hline
        timestamp & 0 & 1 & 2 & 3 & 4 & 5 & 6 & 7 & 8 & \ldots \\ \hline
      \end{tabular}
    \end{figure}
 \end{example}
\end{frame}

\begin{frame}[fragile]{Introduction}
 \begin{example}{}
    \begin{figure}[H]
      \begin{minted}[xleftmargin=80pt, linenos=false, fontsize=\footnotesize]{text}
SELECT t2.id FROM warehouse
WHERE (T as t1; H as h1; T as t2)
FILTER t1[val < 30] AND h1[val < 25]
  AND t2[val > 40] AND t1[id] = h1[id]
  AND h1[id] = t2[id]
WITHIN 10 events
      \end{minted}
    \end{figure}
    \begin{figure}[H]
      \centering
      \begin{tabular}{|c|c|a|c|a|c|c|c|c|a|c}\hline
        type  &$H$&$T$&$H$&$H$&$T$&$H$&$H$&$T$&$T$ & \ldots \\ \hline
        id  & 1 & 1 & 2 & 1 & 2 & 2 & 1 & 1 & 1 & \multirow{2}{*}{\ldots} \\
        val & 50 & 24& 49& 24& 24& 42& 23& 40& 45\\ \hline
        timestamp & 0 & 1 & 2 & 3 & 4 & 5 & 6 & 7 & 8 & \ldots \\ \hline
      \end{tabular}
    \end{figure}
 \end{example}
\end{frame}

\begin{frame}[fragile]{Introduction}
 \begin{example}{}
    \begin{figure}[H]
      \begin{minted}[xleftmargin=80pt, linenos=false, fontsize=\footnotesize]{text}
SELECT t2.id FROM warehouse
WHERE (T as t1; H as h1; T as t2)
FILTER t1[val < 30] AND h1[val < 25]
  AND t2[val > 40] AND t1[id] = h1[id]
  AND h1[id] = t2[id]
WITHIN 10 events
      \end{minted}
    \end{figure}
    \begin{figure}[H]
      \centering
      \begin{tabular}{|c|c|a|c|c|c|c|a|c|a|c}\hline
        type  &$H$&$T$&$H$&$H$&$T$&$H$&$H$&$T$&$T$ & \ldots \\ \hline
        id  & 1 & 1 & 2 & 1 & 2 & 2 & 1 & 1 & 1 & \multirow{2}{*}{\ldots} \\
        val & 50 & 24& 49& 24& 24& 42& 23& 40& 45\\ \hline
        timestamp & 0 & 1 & 2 & 3 & 4 & 5 & 6 & 7 & 8 & \ldots \\ \hline
      \end{tabular}
    \end{figure}
 \end{example}
\end{frame}

%%%%%%%%%%%%%%%%%%%%%

\begin{frame}{Introduction}
 \begin{block}<1->{Partial Matches Problem}
   Under the default \emph{skip-till-any-match} \cite{skip-till-any-match} policy, the set of partial matches grow \emph{exponentially} in $N$.
 \end{block}
 \begin{block}<2->{Related Work}
    CER systems (e.g., Cayuga, Esper, SASE, TESLA/T-REX) still suffer from:
    \begin{itemize}
      \item Overhead \emph{super-linear} in the size of the stream.
      \item Scalability is limited to queries over short time windows.
    \end{itemize}
  \end{block}
\end{frame}

%%%%%%%%%%%%%%%%%%%%%

\begin{frame}{Introduction}
  \begin{block}{CORE}
   \emph{COmplex event Recognition Engine (CORE)\cite{core}}:
   \begin{itemize}
      \item Update in constant time per input.
      \item \emph{Output-linear delay} enumeration.
   \end{itemize}
 \end{block}
\end{frame}

\begin{frame}{Introduction}
  \begin{block}{CORE}
    Downsides of CORE:
   \begin{enumerate}
     \item Limited to \emph{unary} predicates.
     \pause
     \item The enumeration may be exponential in cost.
   \end{enumerate}
 \end{block}
\end{frame}

%%%%%%%%%%%%%%%%%%%%%

\begin{frame}{Introduction}
  \begin{block}{Contributions}
   \begin{itemize}
     \item A distributed framework for CER that circumvents the filtering limitations of many CER systems.
      \begin{itemize}
        \pause
        \item Distributed CER Engine (DCERE)
        \pause
        \item Distributed CORE (DCORE)
      \end{itemize}
     \pause
     \item A novel distributed evaluation algorithm based on CORE.
   \end{itemize}
 \end{block}
\end{frame}

%%%%%%%%%%%%%%%%%%%%%%%%%%%%%%%%%%%%%%%%%%%%%%%%%%%%%%%%%%%%%%%%%%
%%%%%%%%%%%%%%%%%%%%%%%%%%%%%%%%%%%%%%%%%%%%%%%%%%%%%%%%%%%%%%%%%%

\section{Preliminaries}

% \begin{frame}{Preliminaries}
%   \begin{block}{Event}
%     Named mapping between attributes and values.
%   \end{block}
%   \begin{block}{Stream}
%     Possibly infinite sequence $S = t_{0}t_{1}t_{2}\ldots$ of events.
%   \end{block}
%   \begin{block}{Complex Event}
%     A pair $C = ([i,j], D)$ where $i \le j \in \mathbb{N}$ and $D$ is a subset of $\{i, \ldots, j\}$.
%   \end{block}
% \end{frame}

%%%%%%%%%%%%%%%%%%%%%

\begin{frame}{Preliminaries}
  \begin{block}{Complex Event Logic (CEL)}
    Formal logic that is built from the common operators in the literature of CER.
    \begin{equation*}
      \varphi := R    \ | \ \varphi \ \text{AS} \ X    \ | \    \varphi \ \text{FILTER} \ X[P]  \ | \   \varphi \ \text{OR} \ \varphi   \ | \  \varphi ; \varphi    \ | \  \varphi+ \ | \ \pi_{L}(\varphi).
    \end{equation*}
  \end{block}
\end{frame}

%%%%%%%%%%%%%%%%%%%%%

\begin{frame}[fragile]{Preliminaries}
  \begin{block}{Complex Event Query Language (CEQL)}
    CEQL is a practical CER language based on \emph{Complex Event Logic (CEL)}.
  \end{block}
  \begin{block}{Syntax}
    \begin{figure}[H]
      \begin{minted}[xleftmargin=0pt, linenos=false, fontsize=\footnotesize]{text}
        SELECT        [selection strategy] <list of variables>
        FROM          <list of streams>
        WHERE         <CEL formula>
        (PARTITION BY <list of attributes>)?
        (WHITHIN      <time>)?
      \end{minted}
    \end{figure}
  \end{block}
\end{frame}

% \begin{frame}[fragile]{Preliminaries}
%   \begin{block}{Semantics}
%     \begin{figure}[H]
%       \begin{minted}[xleftmargin=0pt, linenos=false, fontsize=\footnotesize]{text}
%         SELECT        [selection strategy] <list of variables>
%         FROM          <list of streams>
%         WHERE         <CEL formula>
%         (PARTITION BY <list of attributes>)?
%         (WHITHIN      <time>)?
%       \end{minted}
%     \end{figure}
%     \begin{enumerate}
%       \item \code{FROM}
%       \item \code{PARTITION BY}
%       \item \code{WHERE}
%       \item \code{SELECT}
%       \item \code{WITHIN}
%     \end{enumerate}
%   \end{block}
% \end{frame}

%%%%%%%%%%%%%%%%%%%%%

\begin{frame}[fragile]{Preliminaries}
  \begin{block}{Complex Event Automaton (CEA)}
     A form of finite automaton that produces complex events.
  \end{block}
  \begin{definition}
  A tuple $\mathcal{A} = (Q, \Delta, q_{0}, F)$ where $Q$ is a finite set of states, $\Delta \subseteq Q \times \textbf{P} \times \{\bullet, \circ\} \times (Q \setminus \{ q_{0} \})$ is a finite transition, $q_{0} \in Q$ is the initial state, and $F \subseteq Q$ is the set of final states. We will denote transitions in $\Delta$ by $q \xrightarrow[]{P/m} q'$.
  \end{definition}
\end{frame}

\begin{frame}[fragile]{Preliminaries}
  \begin{example}
    \begin{figure}[H]
      \begin{minted}[xleftmargin=0pt, linenos=false, fontsize=\footnotesize]{text}
    SELECT *
    FROM warehouse
    WHERE (T as t1; T+ as ts)
    FILTER t1[val < 30]
      AND ts[val > 30]
      \end{minted}
    \end{figure}
    \begin{figure}[H]
      \centering
      \inputtikz{cea}
    \end{figure}
  \end{example}
\end{frame}

%%%%%%%%%%%%%%%%%%%%%%%%%%%%%%%%%%%%%%%%%%%%%%%%%%%%%%%%%%%%%%%%%%
%%%%%%%%%%%%%%%%%%%%%%%%%%%%%%%%%%%%%%%%%%%%%%%%%%%%%%%%%%%%%%%%%%

\section{Distributed CER}

% \begin{frame}{Distributed CER}
%   \begin{block}{Observations}
%     \begin{itemize}
%         \item CORE's sequential evaluation of CEA is \emph{fast}.
%           \begin{itemize}
%             \item Big overhead when distributed.
%           \end{itemize}
%         \pause
%         \item The automaton model is \emph{limited to unary predicates}.
%           \begin{itemize}
%             \item Binary predicates: \code{t1[id] = h1[id]}
%             \item Second-order predicates: the sequence of \code{t[val]} must monotonically increase
%           \end{itemize}
%         \pause
%         \item The \emph{enumeration} process is still slow.
%    \end{itemize}
%   \end{block}
% \end{frame}

\begin{frame}{Distributed CER}
  \begin{block}{Ideas}
    \begin{itemize}
        \item<1> Evaluate CEA sequentially.
        \item<2> Filter complex predicates distributedly.
        \item<3> Enumerate distributedly.
   \end{itemize}
  \end{block}
\end{frame}

\begin{frame}{Distributed CER}
  \begin{block}{Distributed CER framework}
    \begin{figure}[H]
      \centering
      \resizebox{!}{0.7\textheight}{%
        \inputtikz{framework}
      }
    \end{figure}
  \end{block}
\end{frame}

%%%%%%%%%%%%%%%%%%%%%

\begin{frame}{Distributed CER}
  \begin{block}{Distributed CER Engine (DCERE)}
    \begin{figure}[H]
      \centering
      \resizebox{\textwidth}{!}{%
        \inputtikz{dcere}
      }
    \end{figure}
  \end{block}
\end{frame}

\begin{frame}{Distributed CER}
  \begin{block}{Selection Strategies}
    \begin{itemize}
        \item<1> Round Robin (RR)
        \item<2> Power of Two Choices (PoTC)
        \item<3> Exact Search (ES)
        \item<4> Maximal Complex Event Enumeration (MCEE)
   \end{itemize}
  \end{block}
\end{frame}

%%%%%%%%%%%%%%%%%%%%%

\begin{frame}{Distributed CER}
  \begin{block}{Distributed CORE (DCORE)}
    \begin{figure}[H]
      \centering
      \resizebox{\textwidth}{!}{%
        \inputtikz{dcore}
      }
    \end{figure}
  \end{block}
\end{frame}

%%%%%%%%%%%%%%%%%%%%%%%%%%%%%%%%%%%%%%%%%%%%%%%%%%%%%%%%%%%%%%%%%%
%%%%%%%%%%%%%%%%%%%%%%%%%%%%%%%%%%%%%%%%%%%%%%%%%%%%%%%%%%%%%%%%%%

\section{Distributed Evaluation Algorithm}

\begin{frame}{Distributed evaluation algorithm}
  \begin{block}<1->{Goal}
    Distributed evaluation algorithm for CEA $\mathcal{A}$:
   \begin{itemize}
     \item ${\llbracket \mathcal{A} \rrbracket}^{\epsilon}_{j}(S) := \{ C \ | \ C \in {\llbracket \mathcal{A} \rrbracket}^{\epsilon}(S) \land C(end) = j \}$.
   \end{itemize}
  \end{block}

  \begin{block}<2->{Efficiency}
     Under data complexity:
      \begin{itemize}
        \item Update the data structure in constant time per input.
        \item<3-> Output-linear delay enumeration of ${\llbracket \mathcal{A} \rrbracket}^{\epsilon}_{j}(S)$.
        \pause
        \item<4-> Each process filters and enumerates at most $\frac{|{\llbracket \mathcal{A} \rrbracket}^{\epsilon}_{j}(S)|}{|\mathcal{P}|}$ complex events.
      \end{itemize}
  \end{block}
\end{frame}

%%%%%%%%%%%%%%%%%%%%%%

% \begin{frame}{Distributed evaluation algorithm}
%   \begin{block}{The data structure}
%     The data structure is called \emph{timed Enumerable Compact Set (tECS)}.

%     A tECS is a directed acyclic graph $\mathcal{E}$ with two kinds of nodes:

%    \begin{itemize}
%      \item Union nodes \textrm{u}.
%      \item Non-union nodes:
%       \begin{itemize}
%         \item Bottom nodes \textrm{b}.
%         \item Output nodes \textrm{o}.
%       \end{itemize}
%    \end{itemize}

%    Additionally, nodes carry \emph{desceding-paths} attribute.

%    A tECS represents sets of \emph{open complex events}.
%   \end{block}
% \end{frame}

% \begin{frame}{Distributed evaluation algorithm}
%   \begin{block}{Operations on tECS}
%    \begin{itemize}
%      \item \code{$\text{b} \leftarrow \text{new-bottom}(i)$}.
%      \item \code{$o \leftarrow \text{extend}(\textrm{n}, j)$}.
%      \item \code{$\textrm{u} \leftarrow \text{union}(\text{n}_{1},\text{n}_{2})$}.
%    \end{itemize}
%     \begin{figure}[H]
%       \centering
%       \resizebox{0.9\textwidth}{!}{%
%         \begin{subfigure}[b]{0.2\linewidth}
%           \centering
%           \inputtikz{union_a}
%         \end{subfigure}
%         \hfill
%         \begin{subfigure}[b]{0.2\linewidth}
%           \centering
%           \inputtikz{union_b}
%         \end{subfigure}
%         \hfill
%         \begin{subfigure}[b]{0.40\linewidth}
%           \centering
%           \inputtikz{union_c}
%         \end{subfigure}
%         \hfill
%         \begin{subfigure}[b]{0.40\linewidth}
%           \centering
%           \inputtikz{union_d}
%         \end{subfigure}
%       }
%       \caption{Visualisation of the four cases of method union(\textrm{u}). The \textrm{u} are union nodes, where the dashed and bold arrows point to the left and right node, respectively.}
%     \end{figure}
%   \end{block}
% \end{frame}

%%%%%%%%%%%%%%%%%%%%%%%%%

\begin{frame}{Distributed evaluation algorithm}
  \begin{block}{Evaluation algorithm}
    \begin{columns}

      \begin{column}{0.5\textwidth}
        \begin{figure}[H]
          \centering
          \begin{subfigure}[b]{\textwidth}
            \centering
            \inputtikz{cea}
          \end{subfigure}
          \begin{subfigure}[t]{\textwidth}
            \centering
            \inputtikz{stream}
          \end{subfigure}
        \end{figure}
      \end{column}

      \begin{column}{0.5\textwidth}
        \begin{figure}[H]
          \inputtikz{AB+_0}
          \caption*{$S[$0$]$}
        \end{figure}
      \end{column}

    \end{columns}
  \end{block}
\end{frame}

\begin{frame}{Distributed evaluation algorithm}
  \begin{block}{Evaluation algorithm}
    \begin{columns}

      \begin{column}{0.5\textwidth}
        \begin{figure}[H]
          \centering
          \begin{subfigure}[b]{\textwidth}
            \centering
            \inputtikz{cea}
          \end{subfigure}
          \begin{subfigure}[t]{\textwidth}
            \centering
            \inputtikz{stream}
          \end{subfigure}
        \end{figure}
      \end{column}

      \begin{column}{0.5\textwidth}
        \begin{figure}[H]
          \inputtikz{AB+_1}
          \caption*{$S[$1$]$}
        \end{figure}
      \end{column}

    \end{columns}
  \end{block}
\end{frame}

\begin{frame}{Distributed evaluation algorithm}
  \begin{block}{Evaluation algorithm}
    \begin{columns}

      \begin{column}{0.5\textwidth}
        \begin{figure}[H]
          \centering
          \begin{subfigure}[b]{\textwidth}
            \centering
            \inputtikz{cea}
          \end{subfigure}
          \begin{subfigure}[t]{\textwidth}
            \centering
            \inputtikz{stream}
          \end{subfigure}
        \end{figure}
      \end{column}

      \begin{column}{0.5\textwidth}
        \begin{figure}[H]
          \inputtikz{AB+_2}
          \caption*{$S[$2$]$}
        \end{figure}
      \end{column}

    \end{columns}
  \end{block}
\end{frame}

\begin{frame}{Distributed evaluation algorithm}
  \begin{block}{Evaluation algorithm}
    \begin{columns}

      \begin{column}{0.5\textwidth}
        \begin{figure}[H]
          \centering
          \begin{subfigure}[b]{\textwidth}
            \centering
            \inputtikz{cea}
          \end{subfigure}
          \begin{subfigure}[t]{\textwidth}
            \centering
            \inputtikz{stream}
          \end{subfigure}
        \end{figure}
      \end{column}

      \begin{column}{0.5\textwidth}
        \begin{figure}[H]
          \inputtikz{AB+_3}
          \caption*{$S[$3$]$}
        \end{figure}
      \end{column}

    \end{columns}
  \end{block}
\end{frame}

\begin{frame}{Distributed evaluation algorithm}
  \begin{block}{Evaluation algorithm}
    \begin{columns}

      \begin{column}{0.5\textwidth}
        \begin{figure}[H]
          \centering
          \begin{subfigure}[b]{\textwidth}
            \centering
            \inputtikz{cea}
          \end{subfigure}
          \begin{subfigure}[t]{\textwidth}
            \centering
            \inputtikz{stream}
          \end{subfigure}
        \end{figure}
      \end{column}

      \begin{column}{0.5\textwidth}
        \begin{figure}[H]
          \inputtikz{AB+_4}
          \caption*{$S[$4$]$}
        \end{figure}
      \end{column}

    \end{columns}
  \end{block}
\end{frame}

%%%%%%%%%%%%%%%%%%%%%%%%%%%%%%%

\begin{frame}{Distributed evaluation algorithm}
  \begin{block}{Refine and enumeration procedure}
    \begin{figure}[H]
      \inputtikz{AB+_enumeration_0}
      \caption*{Process $0$}
    \end{figure}
  \end{block}
\end{frame}

\begin{frame}{Distributed evaluation algorithm}
  \begin{block}{Refine and enumeration procedure}
    \begin{figure}[H]
      \begin{subfigure}[b]{0.4\linewidth}
        \centering
        \inputtikz{AB+_enumeration_1}
      \end{subfigure}
      \begin{subfigure}[b]{0.4\linewidth}
        \centering
        \inputtikz{AB+_enumeration_2}
      \end{subfigure}
      \caption*{Process $1$}
    \end{figure}
  \end{block}
\end{frame}

\begin{frame}{Distributed evaluation algorithm}
  \begin{block}{Refine and enumeration procedure}
    \begin{figure}[H]
      \inputtikz{AB+_enumeration_3}
      \caption*{Process $2$}
    \end{figure}
  \end{block}
\end{frame}

%%%%%%%%%%%%%%%%%%%%%%%%%%%%%%%%%%%%%%%%%%%%%%%%%%%%%%%%%%%%%%%%%%
%%%%%%%%%%%%%%%%%%%%%%%%%%%%%%%%%%%%%%%%%%%%%%%%%%%%%%%%%%%%%%%%%%

\section{Experiments}

% \begin{frame}{DCORE in a nutshell}
%   \begin{block}{Overview}
%     \begin{itemize}
%       \item To run on the JVM.
%       \item Open source and available at \url{https://github.com/dtim-upc/DCORE}.
%       \item Depends on our fork of CORE \url{https://github.com/dtim-upc/CORE2/tree/distributed_enumeration}.
%     \end{itemize}
%   \end{block}
% \end{frame}

% \begin{frame}{DCORE in a nutshell}
%   \begin{block}{DCORE}
%     \begin{itemize}
%       \item Implemented in Scala 2.12.
%       \item Built on top of Akka Cluster.
%       \item Compiled with Sbt.
%       \item Provided as a white box.
%       \item Rigorously tested.
%     \end{itemize}
%   \end{block}
% \end{frame}

%%%%%%%%%%%%%%%%%%%%%

\begin{frame}[fragile]{Experiments}
  \begin{block}{Goal}
    Compare DCERE, DCORE, and CORE.
  \end{block}
  \begin{block}{Settings}
    Server equipped with a 12 cores Intel i7-8700, 32Gb of RAM, Linux 5.15.
  \end{block}
  \begin{block}{Setup}
    \begin{itemize}
      \item We compare systems with respect to their performance.
      \item Reported numbers are averages taken over three repetitions of each experiment.
      \item Synthetic datasets.
    \end{itemize}
  \end{block}
\end{frame}

\begin{frame}[fragile]{Experiments}
  \begin{block}{Datasets}
    \begin{figure}[H]
      \centering
      \begin{subfigure}[c]{0.32\textwidth}
        \centering
        \begin{minted}[fontsize=\footnotesize, linenos=false, autogobble]{text}
          SELECT *
          FROM S
          WHERE P
          FILTER A[id] = B[id]
          WITHIN 100 events
        \end{minted}
      \end{subfigure}
      \begin{subfigure}[t]{0.32\textwidth}
        \begin{tabular}{l l}
          \hline
          $Q_{1}:$ & $P = A;B;C$ \\
          \hline
          $Q_{2}:$ & $P = A;B+;C$ \\
          \hline
          $Q_{3}:$ & $P = A+;B+;C$ \\
          \hline
        \end{tabular}
      \end{subfigure}
      \hfill
      \begin{subfigure}[c]{0.32\textwidth}
        \centering
        \begin{tabular}{l c}
          \hline
          $S_{1}:$ & A,A$\ldots$B,B$\ldots$C\\
          \hline
          $S_{2}:$ & A,B,B,B$\ldots$C\\
          \hline
          $S_{3}:$ & A,B,A,B$\ldots$C\\
          \hline
        \end{tabular}
      \end{subfigure}
    \end{figure}
  \end{block}
\end{frame}

%%%%%%%%%%%%%%%%%%%%%

\begin{frame}{Experiments}
  \begin{block}{On the evaluation of complex predicates}
    \begin{figure}[H]
        \centering
        \begin{subfigure}[b]{0.40\textwidth}
            \centering
            \includegraphics[width=\textwidth]{experiment_1_chart_1}
            \tiny (a) $Q_{1}$
        \end{subfigure}
        \begin{subfigure}[b]{0.40\textwidth}
            \centering
            \includegraphics[width=\textwidth]{experiment_1_chart_2}
            \tiny (b) $Q_{2}$
        \end{subfigure}
        \begin{center}
          \begin{subfigure}[b]{0.40\textwidth}
              \centering
              \includegraphics[width=\textwidth]{experiment_1_chart_3}
              \tiny (c) $Q_{3}$
          \end{subfigure}
        \end{center}
    \end{figure}
  \end{block}
\end{frame}

%%%%%%%%%%%%%%%%%%%%%%

\begin{frame}{Experiments}
  \begin{block}{On the scalability of the framework}
    \begin{figure}[H]
        \centering
        \begin{subfigure}[b]{0.49\textwidth}
            \centering
            \includegraphics[width=\textwidth]{experiment_3_chart_1}
            \tiny (a) $1024$ complex events
        \end{subfigure}
        \begin{subfigure}[b]{0.49\textwidth}
            \centering
            \includegraphics[width=\textwidth]{experiment_3_chart_2}
            \tiny (b) $2048$ complex events
        \end{subfigure}
    \end{figure}
  \end{block}
\end{frame}

%%%%%%%%%%%%%%%%%%%%%%

\begin{frame}{Experiments}
  \begin{block}{On DCORE under heavy loads}
    \begin{figure}[H]
        \centering
        \begin{subfigure}[b]{0.4\textwidth}
            \centering
            \includegraphics[width=\textwidth]{experiment_4_chart_1}
            \tiny (a) $Q_{1}$
        \end{subfigure}
        \begin{subfigure}[b]{0.4\textwidth}
            \centering
            \includegraphics[width=\textwidth]{experiment_4_chart_2}
            \tiny (b) $Q_{2}$
        \end{subfigure}
        \begin{center}
        \begin{subfigure}[b]{0.4\textwidth}
            \centering
            \includegraphics[width=\textwidth]{experiment_4_chart_3}
            \tiny (c) $Q_{3}$
        \end{subfigure}
        \end{center}
    \end{figure}
  \end{block}
\end{frame}


%%%%%%%%%%%%%%%%%%%%%%%%%%%%%%%%%%%%%%%%%%%%%%%%%%%%%%%%%%%%%%%%%%
%%%%%%%%%%%%%%%%%%%%%%%%%%%%%%%%%%%%%%%%%%%%%%%%%%%%%%%%%%%%%%%%%%

\section{Conclusions and Future Work}

\begin{frame}{Conclusions \& future work}
  \begin{block}{Conclusions}
   \begin{itemize}
     \item We presented:
        \begin{itemize}
          \item A framework for distributed CER.
          \pause
          \item Two implementations: DCERE and DCORE.
          \pause
          \item A distributed evaluation algorithm for DCORE.
        \end{itemize}
     \pause
     \item We showed that both implementations of our framework outperform its competitors on queries with complex predicates.
   \end{itemize}
  \end{block}
\end{frame}

%%%%%%%%%%%%%%%%%%%%%%

\begin{frame}{Conclusions \& future work}
  \begin{block}{Future Work}
   \begin{itemize}
     \item Extend our framework with a generic rewrite and refine algorithms.
     \pause
     \item Extend our distributed evaluation algorithm to take into account time windows.
   \end{itemize}
  \end{block}
\end{frame}

%%%%%%%%%%%%%%%%%%%%%%

\begin{frame}[c]{ }
  \centering
  \huge Questions ?
\end{frame}

%%%%%%%%%%%%%%%%%%%%%%%%%%%%%%%%%%%%%%%%%%%%%%%%%%%%%%%%%%%%%%%%%%
%%%%%%%%%%%%%%%%%%%%%%%%%%%%%%%%%%%%%%%%%%%%%%%%%%%%%%%%%%%%%%%%%%

\begin{frame}[allowframebreaks]
  \frametitle{Bibliography}
  \bibliographystyle{unsrt}
  \bibliography{bibliography}
\end{frame}

\end{document}

